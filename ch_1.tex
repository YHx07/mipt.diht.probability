\section{Глава 1. Вероятностная мера, функция распределения.}
\begin{definition}
    Под $(\Omega, \F, \P)$ будем понимать вероятностное пространство, где:
    \begin{enumerate}
    	\item $\Omega$~--- пространство элементарных исходов;
    	\item $\F$~--- $\sigma$-алгебра на $\Omega$;
    	\item{$\P: \F \rightarrow [0, 1]$~--- вероятностная мера, причем:
    		\begin{enumerate}
    			\item[a)] $\P (\Omega) = 1$;
    			\item[b)] $\P$~--- $\sigma$-аддитивна, то есть $\forall \{A_n\}_{n = 1}^{+\infty} \in \F$, причем $A_n \cap A_m = \varnothing$ при $n \neq m$: $\P \left( \bigsqcup\limits_{n=1}^{+\infty} A_n \right) = \sum \limits_{n = 1}^{+\infty} \P (A_n)$.
    		\end{enumerate}
    	}
    \end{enumerate}
\end{definition}

\begin{definition}
	Последовательность $\{A_n\}$ убывает к $A$, если $\forall n \hookrightarrow A_{n} \supseteq A_{n+1}$ и   $A = \bigcap\limits_{n = 1}^{+\infty} A_n$.
	Последовательность $\{A_n\}$ возрастает к $A$, если $\forall n \hookrightarrow A_{n} \subseteq A_{n+1}$ и   $A = \bigcup\limits_{n = 1}^{+\infty} A_n$.
\end{definition}

\begin{theorem}[О непрерывности вероятностной меры][б/д]~

	Пусть $(\Omega, \F )$~--- измеримое пространство и на нем определена функция $\P: \F \rightarrow [0,1]$, удовлетворяющая следующим свойствам: 
	\begin{enumerate}
	    \item $\P (\Omega) = 1$;
	    \item  $\P$~--- конечно аддитивная.
	\end{enumerate} 
	Тогда следующие утверждения эквивалентны:
	\begin{enumerate}
		\item $\P$~--- вероятностная мера;
		\item $\forall A_n \downarrow A \hookrightarrow \P (A_n) \rightarrow \P(A)$ (непрерывность снизу);
		\item $\forall A_n \uparrow A \hookrightarrow \P (A_n) \rightarrow \P(A)$ (непрерывность сверху);
		\item  $\forall A_n \downarrow \varnothing \hookrightarrow \P (A_n) \rightarrow 0$ (непрерывность в нуле).
	\end{enumerate}
\end{theorem}

\begin{theorem}[Каратеодори][б/д]~

	Пусть $\Omega$~--- некое множество, $\mathcal{A}$~--- алгебра на $\Omega$ и $\P_\sigma$~--- вероятностная мера на $(\Omega, \mathcal{A})$. 
	
	Тогда существует единственная вероятностная мера на $\big(\Omega, \sigma(\mathcal{A}) \big)$, являющаяся продолжением $\P_\sigma$, то есть  $ \forall A \in \mathcal{A} \hookrightarrow \P_\sigma(A) = \P(A)$.
\end{theorem}

%%-----------------------------------------------------------------------%%

\subsection{Функция распределения}

Рассмотрим измеримое пространство $\big(\R, \B(\R)\big)$ и вероятностную меру $\P$~на~нем.

\begin{definition}
	Функция $F(x), x \in \R$, заданная по правилу $F(x) = \P \big( (-\infty, x] \big)$~--- функция распределения вероятностной меры  $\P$.
\end{definition}

\begin{lemma}[свойства функции распределения]~

	Пусть $F(x)$~--- функция распределения, тогда
	\begin{enumerate}
		\item $F(x)$ не убывает;
		\item $\lim\limits_{x \rightarrow +\infty} F(x) = 1$; $\lim\limits_{x \rightarrow -\infty} F(x) = 0$;
		\item $F(x)$ непрерывна справа.
	\end{enumerate}
	\begin{proof}
		Пусть $y \geqslant x$, тогда $F(y) - F(x) = \P \big( (-\infty, y] \big) - \P \big( (-\infty, x] \big) = \P \big( (x, y] \big) \geqslant 0$, следовательно, $F(x)$ неубывает.
		
		Пусть $x_n \rightarrow -\infty$ при $n \rightarrow +\infty$, тогда $(-\infty, x_n] \rightarrow \varnothing$, следовательно, $F(x_n) = \P \big( (-\infty, x_n] \big) \underset{n \rightarrow +\infty}{\longrightarrow} 0$ по теореме о непрерывности вероятностной меры. 
		
		Пусть $x_n \rightarrow +\infty$ при $n \rightarrow +\infty$, тогда $(-\infty, x_n] \rightarrow \R$, следовательно, $F(x_n) = \P \big( (-\infty, x_n] \big) \underset{n \rightarrow +\infty}{\longrightarrow} \P(\R) = 1$.
				
		Пусть $x_n \downarrow x$, тогда $(-\infty, x_n] \downarrow (-\infty, x]$, отсюда по теореме о непрерывности вероятностной меры вытекает, что $F(x_n) = \P \big( (-\infty, x_n] \big) \underset{n \rightarrow +\infty}{\longrightarrow} \P \big( (-\infty, x] \big) =~F(x)$.
	\end{proof}
\end{lemma}

\begin{property}
	Функция распределения имеет предел слева $\forall x \in \R$, при этом число точек разрыва не более, чем счетно.
	\begin{proof}
		Пусть $x_n \rightarrow x - 0$~--- возрастающая последовательность, тогда $F(x_n) = \P \big( (-\infty, x_n] \big) \underset{n \rightarrow +\infty}{\longrightarrow} \P \big( (-\infty, x) \big) = F(x - 0)$. 
		
		Каждая точка разрыва~--- скачок функции распределения, каждому скачку сопоставим $[F(x-0), F(x)]$, а этому отрезку в свою очередь сопоставим некую рациональную точку, которая лежит в $(F(x-0), F(x))$. Следовательно каждому скачку мы сопоставили точку из $\mathbb{Q}$, а так как $\mathbb{Q}$ счетно, то число разрывов не более, чем счетно.
	\end{proof}
\end{property}

\begin{definition}
	Функция $F(x)$, удовлетворяющая свойствам 1) -- 3) из леммы, называется функцией распределения на $\R$.
\end{definition}

\begin{theorem}[О взаимно однозначном соответствии между вероятностной мерой и функцией распределения на $\R$]~

	Пусть $F(x)$~--- функция распределения на $\R$.
	
	Тогда существует единственная вероятностная мера $\P$ на $\big(\R, \B(\R) \big)$ такая, что $F(x)$ является ее функцией распределения, то есть $F(x) = \P \big( ( -\infty, x] \big)$.
	\begin{proof}
		Рассмотрим полукольцо $ S = \big\{ (a, b] \big\}$ на $\R$. Определим $\sigma$-аддитивную вероятностную меру $\P \big( (a, b] \big) = F(b) - F(a)$, а по теореме Каратеодори $\P$ единственным образом продолжается на всю $\sigma$-алгебру $\B(\R)$.
	\end{proof}
\end{theorem}

\subsection{Классификация вероятностных мер и функций распределения на прямой}
\subsubsection*{\ovalbox{1}~~Дискретное распределение}
Пусть $\mathscr{X} \subseteq \R$ не более, чем счетно.
\begin{definition}
	Вероятностная мера $\P$ на $\big( \R, \B(\R) \big)$, удовлетворяющая свойству $\P(\R \backslash \mathscr{X}) = 0$, называется дискретной вероятностной мерой на $\mathscr{X}$, ее функция распределения также называется дискретной.
\end{definition}

Рассмотрим $\mathscr{X} = \{ x_k \}$, положим $p_k = \P \big( \{x_k \} \big)$, тогда $\P( \mathscr{X} ) = 1 = \sum\limits_{k} \P ( x_k )$.

\begin{definition}
	Набор чисел $\{ p_k \}$~--- распределение вероятностей на $\mathscr{X}$.
\end{definition}

\subsubsection*{\ovalbox{2}~~Абсолютно непрерывное распределение}
\begin{definition}
	Пусть $F(x)$~--- функция распределения вероятностной меры $\P$ на $\R$, причем $\forall x \in \R \hookrightarrow F(x) = \int\limits_{-\infty}^{x} p(t) \, d t$, где  $p(t) \geqslant 0$, а $\int\limits_{-\infty}^{+\infty} p(t) \, d t = 1$. 
	
	Тогда $\P$ абсолютно непрерывна, $F(x)$ также называется абсолютно непрерывной, а $p(t)$~--- плотность функции распределения $F(x)$. Причем $p(t)$ определена однозначно, кроме множества меры нуль.
	
	Из формулы Ньютона-Лейбница: если $F(x)$~--- дифференцируема, то $p(x) = F'(x)$.
	
	Если $p(x)$~--- плотность функции распределения $F(x)$, то $\forall B \in \B(\R) \hookrightarrow \P(B) = \int\limits_{B} p(x)dx$.
\end{definition}

\noindent {\bfseries Примеры:}
\begin{enumerate}
	\item{Равномерное распределение $R[a,b]$ 
		$$p(x) = \frac{1}{b -a } \cdot I \big( x \in [a,b] \big). $$
	}
		
	\item{Нормальное (гауссовское) распределение $N(a, \sigma^2)$
		$$p(x) = \frac{1}{\sqrt{2 \pi \sigma^2}} \cdot \exp \left[ -\frac{(x - a)^2}{2 \sigma^2} \right]. $$
	}
		
	\item{Экспоненциальное распределение $\text{Exp}(\alpha)$
	    $$p(x) = \alpha e^{-\alpha x} \cdot I(x > 0).$$
	}
	
	\item{Распределение Коши $\text{Cauchy}(\theta)$
		$$p(x) = \frac{\theta}{\pi \left(x^2 + \theta^2 \right)}.$$
	}
		
	\item{Гамма распределение $\Gamma(\alpha, \gamma)$
		$$p(x) = \frac{x^{\alpha - 1} \gamma^{\alpha}}{\Gamma(\alpha)} \cdot e^{-\gamma x} \cdot I(x > 0).$$
	}
\end{enumerate}

\begin{definition}
	$\Gamma(\alpha) = \int\limits_0^{+\infty} x^{\alpha - 1} e^{-x} \, d x$, причем $\forall n \in \N \hookrightarrow \Gamma(n) = (n - 1)!$, $\forall \lambda \in \R \hookrightarrow \Gamma(\lambda \pm 1) = \lambda \Gamma(\lambda)$, а $\Gamma\left(\dfrac{1}{2}\right) = \sqrt{\pi}$.
\end{definition}

\subsubsection*{\ovalbox{3}~~Сингулярные распределения}
\begin{definition}
	Пусть $F(x)$~--- функция распределения на $\R$. Точка $x_0 \in \R$ называется точкой роста $F(x)$, если $\forall \varepsilon > 0 \hookrightarrow F(x_0 + \varepsilon) - F(x_0 - \varepsilon) > 0$.
\end{definition}

\begin{definition}
	Функция распределения называется сингулярной, если она непрерывна и множество ее точек роста имеет Лебегову меру нуль. Например, функция Кантора.
\end{definition}
\begin{theorem}[Лебега о функции распределения][б/д]~

	Пусть $F(x)$~--- функция распределения на $\R$.
	
	Тогда существуют единственные $\alpha_1, \alpha_2$ и $\alpha_3$, $\alpha_i \geqslant 0, \alpha_1 + \alpha_2 + \alpha_3 = 1$ и функции распределения $F_1(x), F_2(x)$ и $F_3(x)$ такие, что $F(x) = \alpha_1 F_1(x) + \alpha_2 F_2(x) + \alpha_3 F_3(x)$, где $F_1(x)$~--- дискретная функция распределения, $F_2(x)$~--- абсолютно непрерывная, а $F_3(x)$~--- сингулярная. 
\end{theorem}

\subsection{Вероятностная мера в $\big( \R^n, \B(\R^n) \big)$}
\begin{definition}
	Пусть $\P$~--- вероятностная мера в $\big( \R^n, \B(\R^n) \big)$, где $\B(\R^n) = \sigma\{B_1 \times B_2 \times \dots \times B_n,\ B_i \in \B(\R),\ i \leqslant n \}$, тогда функция $F(\vec x) = \P\big((-\infty; x_1] \times \ldots \times (-\infty, x_n]\big)$ называется функцией распределения вероятностной меры $\P$ в $\R^n$.
\end{definition}

\begin{note}
	Пусть $\vec x^{(k)} = \left(x_1^{(k)}, \ldots, x_n^{(k)}\right) \in \R^n$. Будем писать $\vec x^{(k)} \downarrow \vec x$, если $\forall i, k \hookrightarrow x_i^{(k)} \geqslant x_i^{(k + 1)}$ и $x_i^{(k)} \underset{k \rightarrow +\infty}{\longrightarrow} x_i$.
\end{note}

\begin{lemma}[Свойства многомерной функции распределения]~

	Пусть $F(\vec x)$~--- функция распределения вероятностной меры в $\R^n$.
	
	Тогда для нее верно:
	\begin{enumerate}
		\item Если $\vec x^{(k)} \downarrow \vec x$, то $F\big(\vec x^{(k)}\big) \rightarrow F\big(\vec x\big), k \rightarrow +\infty$;
		\item $\lim\limits_{\forall i : x_i \rightarrow +\infty} F(\vec x) = 1;\; \lim\limits_{\exists i : x_i \rightarrow -\infty} F(\vec x) = 0$;
		\item $\forall a_1 < b_1, a_2 < b_2, \ldots,\;\; \Delta_{a_1b_1}^1 \ldots \Delta_{a_nb_n}^nF(x) > 0 :$, где \\
		$\Delta_{a_ib_i}^i F(\vec x) = F(x_1, \ldots, x_{i - 1}, b_i, x_{i + 1}, \ldots x_n) - F(x_1, \ldots, x_{i - 1}, a_i, x_{i + 1}, \ldots, x_n)$.
	\end{enumerate}
	\begin{proof}
		Первое свойство следует из непрерывности вероятностной меры, так как $$\bigtimes\limits_{i = 1}^n\big(-\infty, x_i^{(k)}\big] \downarrow \bigtimes\limits_{i = 1}^n \big(-\infty, x_i\big].$$
		Для доказательства второго пункта рассмотрим
		$$B_m = \bigtimes\limits_{i = 1}^n\big(-\infty, \inf\limits_{k \geqslant m} x_i^{(k)}\big].$$
		Если $\forall i : x_i^k \rightarrow +\infty$, то $B_m \rightarrow \R^n$, $\P(B_m) \rightarrow \P(\R^n) = 1$. А если $\exists i : x_i^{(k)} \rightarrow - \infty$, то $B_m \rightarrow \varnothing, \P(B_n) \rightarrow 0$.
		
		Не трудно понять, что 
		$$\Delta_{a_1b_1}^1 \ldots \Delta_{a_nb_n}^nF(x) = \P\big((a_1, b_1] \times \ldots \times (a_n, b_n]\big),$$
		откуда следует утверждение третьего пункта леммы. Так, например,
		$$\Delta_{a_1b_1}^1\Delta_{a_2,b_2}^2F(x) = F(b_1,b_2) - F(a_1,b_2) - (F(b_1,a_2) - F(a_1,a_2)).$$
	\end{proof}
\end{lemma}

\begin{theorem}[О взаимооднозначном соответствии вероятностной меры и функции распределения в $\R^n$][б/д]~

	Если функция $F(\vec x),\; \vec x \in \R^n$ удовлетворяет свойствам из леммы, то существует единственная $\P$ на $\big(\R^n, \B(\R^n) \big)$, для которой $F(\vec x)$ является функцией распределения.
\end{theorem}

\begin{note}
	Почему нельзя заменить свойство 3) на монотонность на любом компакте? 
	\begin{proof}
		Пусть $F(x_1, x_2) = \max(x_1, x_2)$ на $[0, 1]^2$, но тогда $-1 = \Delta_{0; 1}^1 \Delta_{0; 1}^2 F(x_1, x_2) \neq \P([0, 1]^2) = 1$. Следовательно, $F(x)$ не функция распределения.
	\end{proof}
\end{note}

\begin{definition}
	Функция $F(\vec x)$, удовлетворяющая условиями из леммы называется функцией распределения в $\R^n$.
\end{definition}

\begin{definition}
    Маргинальной функцией распределения $i$-й компоненты функции распределения $F(\vec x)$ называется $F_i (x_i) = F(+\infty, \dots, +\infty, x_i, +\infty, \dots, +\infty)$.
\end{definition}

\subsection{Многомерная плотность вероятности}
\begin{definition}
	Если 
	$$ F(x_1, \ldots, x_n) = \int\limits_{-\infty}^{x_1}\dots\int\limits_{-\infty}^{x_n} p(y_1, \ldots, y_n)dy_1\ldots dy_n,\; p(x_1, \ldots, x_n) \geqslant 0, $$
	$$ \idotsint\limits_{\R^n} p(t_1, \dots, t_n) \,dt_1 \dots dt_n = 1,$$
	то $p(x_1, \ldots, x_n)$ называется $n$-мерной плотностью вероятности.
	
	Если дифференцируема $F(x_1, \dots, x_n)$, то
	$$ p(x_1, \ldots, x_n) = \frac{\partial^n}{\partial x_1\ldots\partial x_n}F(x_1, \ldots, x_n). $$
\end{definition}