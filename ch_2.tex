\section{Глава 2.}
\subsection{Случайные величины}
\begin{definition}
	Пусть $\big(\Omega, \F \big)$ и $\big( E, \mathcal{E} \big)$~--- два измеримых пространства. Отображение $X: \Omega \rightarrow E$, такое что $\forall B \in \mathcal{E} \hookrightarrow X^{-1} \in \F$  называется случайным элементом, так же его называются $\F$-измеримым или $\F|\mathcal{E}$-измеримым. \\
	Если $(E, \mathcal{E}) = \big(\R, \B(\R)\big)$, то это случайная величина. \\
	Если $(E, \mathcal{E}) = \big(\R^n, \B(\R^n)\big)$, то это случайный вектор.
\end{definition}

\begin{definition}
	Функция $\varphi: \R^n \rightarrow \R^m$~--- борелевская, если $\forall B \in \B(\R^m) \hookrightarrow \varphi^{-1}(B) \in \B(\R^n)$. 
\end{definition}

\begin{statement}
	Любая непрерывная и кусочно-непрерывная функция~--- борелевская.
\end{statement}

\begin{theorem}[критерий измеримости][б/д]
	Пусть $\big(\Omega, \F\big), \big(E, \mathcal{E}\big)$~--- два измеримых пространства, $X: \Omega \rightarrow E$~--- случайный элемент тогда, и только тогда, когда существует система событий $\M \subseteq \mathcal{E}$, такая что $\sigma(\M) = \mathcal{E}$ и $\forall B \in \M \hookrightarrow X^{-1}(B) \in \F$.
\end{theorem}

\begin{lemma}
	Пусть $\vec\xi = (\xi_1, \ldots, \xi_n)$~--- случайный вектор, $\varphi: \R^n \rightarrow \R^m$~--- борелевская функция, тогда $\varphi(\vec\xi)$~--- случайный вектор.
	\begin{proof}
		Пусть $B \in \B(\R^m)$. Тогда 
		$$ (\varphi(\vec\xi))^{-1}(B) = \big\{\omega : \varphi(\vec\xi(\omega)) \in B\big\} = \big\{\omega : \vec\xi(\omega) \in \varphi^{-1}(B) \subseteq \B(\R^n) \big\} \in \F. $$ Так как выполняется $\forall B$, то $\varphi(\xi)$~--- случайный вектор.
	\end{proof}
\end{lemma}

\begin{lemma}
	$\vec \xi = (\xi_1, \ldots, \xi_n)$~--- случайный вектор тогда, и только тогда, когда $\forall i : \xi_i$~--- случайная величина.
	\begin{proof}
		\emph{Необходимость.} $\varphi(x_1, \ldots, x_n) = x_i$~--- непрерывная функция, значит борелевская, следовательно, по предыдущей лемме $\xi_i$~--- случайная величина. \\
		\emph{Достаточность.} $\B(\R^n) = \sigma\big(B_1 \times \ldots \times B_n, B_i \in \B(\R)\big)$, поэтому $\vec\xi^{-1}(B_1 \times \ldots \times B_n) = \big\{\omega : \vec\xi(\omega) \in B_1 \times \ldots \times B_n\big\} = \big\{\omega : \xi_1(\omega) \in B_1, \ldots, \xi_n(\omega) \in B_n\big\} = \bigcap\limits_{i = 1}^n\big\{\omega: \xi(\omega) \in B_i \big\} = \bigcap\limits_{i = 1}^n \xi_i^{-1}(B_i) \in \F$, значит, по критерию измеримости, $\vec \xi$~--- случайный вектор.
	\end{proof}
\end{lemma}

\begin{consequence}
	Пусть $\xi$, $\eta$~--- случайные величины, $c \in \R$,тогда $\xi + \eta$, $\xi - \eta$, $c\xi$, $\xi \cdot \eta$ и $\xi / \eta$, если $\forall \omega \in \Omega : \eta \neq 0$, тоже случайные величины.
\end{consequence}

\begin{lemma}[О пределах случайной величины]
	Пусть $\{ \xi_n \}_{n \in \N}$ последовательность случайных величин, тогда, если пределы $\overline{\lim}~\xi_n$, $\underline{\lim}~\xi_n$, $\inf\xi_n$, $\sup\xi_n$ существуют, они являются случайными величинами.
	\begin{proof}
		$\big\{\omega : \sup \xi_n \leqslant x \big\} = \bigcap\limits_{n = 1}^{+\infty}\big\{\omega : \xi_n(\omega) \leqslant x \big\} \in \F$. По критерию измеримости, так как $\sigma\big(x : (-\infty, x]\big) = \B(\R)$, мы доказали, что $\sup\xi_n$~--- случайная величина. Аналогично, $\big\{\omega : \inf \xi_n \geqslant x \big\} = \bigcup\limits_{k = 1}^{+\infty}\big\{\omega : \xi_n(\omega) \geqslant x \big\} \in \F$, так как $\sigma\big((x, +\infty)\big) = \B(\R)$, по критерию измеримости $\inf\xi_n$~--- случайная величина. Отсюда $\overline{\lim}~\xi_n = \inf\limits_n\sup\limits_{m \geqslant n}~\xi_m$ и $\underline{\lim}~\xi_n = \sup\limits_n\inf\limits_{m \geqslant n}~\xi_m$ тоже случайные величины.
	\end{proof}
\end{lemma}

\begin{consequence}
	Пусть $\xi = \lim \xi_n$ и предел существует $\forall \omega \in \Omega$, тогда $\xi$~--- случайная величина.
	\begin{proof}
	$\xi = \lim\limits_n \xi_n = \overline{\lim}_n \xi_n = \underline{\lim}_n \xi_n$. Тогда $\xi$~--- случайная величина.
	\end{proof}
\end{consequence}

\subsection{Характеристики случайных величин и векторов}
\subsubsection*{\ovalbox{1} Распределение случайной величины}
\begin{definition}
	Пусть $\big(\Omega, \F, \P \big)$~--- вероятностное пространство, $\xi$~--- случайная величина (вектор) на нем. Распределением случайной величины называется вероятностная мера $\P_\xi$ на $\big(\R, \B(\R)\big)$ $\big((\R^n, \B(\R^n))\big)$, заданная по правилу $\P_\xi(B) = \P(\xi \in B)$, $B \in \B(\R)$ ($B \in \B(\R^n)$).
\end{definition}

\subsubsection*{\ovalbox{2} Функция распределения случайной величины}
\begin{definition}
	Пусть $\big(\Omega, \F, \P\big)$~--- вероятностное пространство, $\xi$~--- случайная величина (вектор) на нем. Функцией распределения $\xi$ называется $F_\xi(x) = \P(\xi \leqslant x)$ ($F_\xi(\vec x) = \P(\xi_1 \leqslant x_1, \ldots, \xi_n \leqslant x_n)$).
\end{definition}

\subsubsection*{\ovalbox{3} Дискретность и непрерывность}
\begin{definition}
	Случайная величина называется дискретной, если ее распределение дискретно.
\end{definition}

\begin{definition}
	Случайная величина называется абсолютно непрерывной, если ее распределение абсолютно непрерывно, то есть $F_\xi(x) = \int\limits_{-\infty}^xp_\xi(y)dy,\; p_\xi(y) \geqslant 0$~--- плотность случайной величины $\xi$.
\end{definition}

\begin{definition}
    Случайная величина называется сингулярной, если её распределение сингулярно.
\end{definition}

\subsubsection*{\ovalbox{4} Сигма-алгебра, порожденная случайной величиной}
\begin{definition}
	Пусть $\xi = \xi(\omega)$~--- случайная величина (вектор) на $\big(\Omega, \F, \P\big)$, тогда $\sigma$-алгеброй $\F_\xi$, порожденной случайной величиной $\xi$, называется $\F_\xi = \big\{\xi^{-1}(B), B \in \B(\R)\ (\B(\R^n)) \big\}$.
\end{definition}

\setcounter{property}{0}

\begin{definition}
	Пусть $\xi, \eta$~--- случайные величины. Тогда $\eta$ является $\F_\xi$-измеримой, если $\F_\eta \subset \F_\xi$.
\end{definition}

\begin{example}
	Пусть $f$~--- борелевская, $\eta = f(\xi)$. Тогда $\eta$~--- $\F_\xi$-измерима.
	\begin{proof}
		$\{\eta \in B\} \in \F_\xi$, где $B \in \B(\R)$, значит $\{\eta \in B\} = \{\xi \in f^{-1}(B) \in \B(\R)\} \in~\F_\xi$
	\end{proof}
\end{example}

\begin{theorem}[][б/д]
	Случайная величина $\eta$~--- $\F_\xi$-измерима, тогда и только тогда, когда существует борелевская $\varphi$, такая что $\forall \omega \in \Omega \hookrightarrow \eta(\omega) = \varphi(\xi(\omega))$ почти наверное, то есть $\P\big(\eta = \varphi(\xi)\big) = 1$.
\end{theorem}

\subsection{Независимость случайных величин}
%\begin{statement}
%    Случайные величины независимы тогда, и только тогда, когда порождаемые ими $\sigma$-алгебры независимы.
%\end{statement}

\begin{definition}
	Системы множеств $\F$ и $\G$ независимы, если $\forall A \in \F,\, B \in \G \hookrightarrow \P(A \cap B) = \P(A)\cdot\P(B)$.
\end{definition}

\begin{definition}
	Пусть $\xi$ и $\eta$~--- случайные величины, тогда $\xi$ и $\eta$ независимы, если $\forall B_1, B_2 \in \B(\R) \hookrightarrow  \P(\xi \in B_1, \eta \in B_2) = \P(\xi \in B_1)\cdot\P(\eta \in B_2)$.
\end{definition}

\begin{definition}
	Случайные величины $\{\xi_i\}_{i = 1}^\infty$ независимы (в совокупности), если для любого конечного набора индексов $\alpha_1, \ldots, \alpha_n \hookrightarrow \P(\xi_{\alpha_1} \in B_1, \ldots, \xi_{\alpha_n} \in B_n) = \prod\limits_{i = 1}^n\P(\xi_{\alpha_i} \in B_i),\; B_i \in \B(\R), i = 1, \ldots, n$.
\end{definition}

\begin{theorem}[Критерий независимости в терминах функции распределения]
	Случайные величины $\{\xi_i\}_{i = 1}^n$ независимы в совокупности тогда, и только тогда, когда $\forall x_1, \ldots, x_n \in \R \hookrightarrow \P(\xi_1 \leqslant x_1, \ldots, \xi_n \leqslant x_n) = \prod\limits_{i = 1}^n\P(\xi_i \leqslant x_i)$.
	\begin{proof}
		$\Rightarrow$. Возьмем в качестве $B_i = (-\infty, x_i]$.
		
		$\Leftarrow$. Не доказываем.
	\end{proof}
\end{theorem}

\begin{theorem}
	Пусть $(\xi_1, \ldots, \xi_n)$~--- независимые в совокупности случайные векторы, $\xi_i$ имеет размерность $n_i$. Пусть $f_i: \R^{n_i} \rightarrow \R^{k_i}$~--- борелевские функции. Тогда величины $f_1(\xi_1), \ldots, f_n(\xi_n)$~--- независимые в совокупности.
	\begin{proof}
		Обозначим $\eta_i = f_i(\xi_i) \Rightarrow \eta_i$~--- $\F_{\xi_i}$-измеримая. По условию $\{\F_{\xi_i}\}_{i = 1}^n$~--- независимые $\sigma$-алгебры, следовательно $\{\F_{\eta_i}\}$ независимы, т.к. $\forall i : \F_{\eta_i} \subset \F_{\xi_i}$, значит по определению $\{\eta_i\}$ независимы в совокупности.
	\end{proof}
\end{theorem}