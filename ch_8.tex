\section{Глава 8.}
\subsection{Гауссовские случайные векторы}
	\begin{definition}[1]
		Случайный вектор $\vec{\xi} \sim N(\vec{m}, \Sigma)$~--- гауссовский, если его характеристическая функция $\varphi_{\vec{\xi}}(\vec{t}) = \E e^{i(\vec{t},\vec{\xi})} = \exp\left(i(\vec{m}, \vec{t}) - \frac{1}{2}(\Sigma \vec{t}, \vec{t})\right), ~ \vec{m} \in \R^n, ~\Sigma \in \operatorname{Mat}_{n \times n}$~--- симметричная неотрицательно определённая матрица.
	\end{definition}

	\begin{definition}[2]
		Случайный вектор $\vec{\xi}  = (\xi_1, \dots, \xi_n)^T$~--- гауссовский, если он представляется в следующем виде: $\vec{\xi} = A\vec{\eta} + \vec{B}$, где $\vec{B} \in \R^n, ~A\in \operatorname{Mat}_{n\times m}$ и $\vec{\eta} = (\eta_1, \ldots, \eta_m)^T$~--- независимые и $\sim N(0, 1)$.
	\end{definition}

	\begin{definition}[3]
		Случайный вектор $\vec{\xi}$~--- гауссовский, если \(\forall \lambda \in \R^n\) случайная величина \((\vec{\lambda}, \vec{\xi})\) имеет нормальное распределение.
	\end{definition}

	\begin{theorem}
		Предыдущие 3 определения эквивалентны.
	\end{theorem}
	\begin{proof}
		\begin{enumerate}
			\item Опр 1 $\Rightarrow$ Опр 2. Пусть $\varphi_{\vec{\xi}}(\vec{t}) = Ee^{i(\vec{t},\vec{\xi})} = e^{i(\vec{t}, \vec{m}) - \frac{1}{2}(R\vec{t}, \vec{t}))}\). Так как матрица \(R\)~--- симметричная и неотрицательно определённая, то \(\exists S\)~---  ортогональная, такая что \(S^T RS = D = \left(
            \begin{array}{ccccc}
                 d_1                              \\
                           &\ddots &  & \text{\huge0}\\
              &               & d_k               \\
              & \text{\huge0} &   & 0            \\
              &               &   &   & 0
            \end{array}
            \right), d_i > 0.$
        
            Определим $\tilde{D} = \left(
            \begin{array}{ccccc}
                 \frac{1}{\sqrt{d_1}}                               \\
                           &\ddots &  & \text{\huge0}               \\
              &               & \frac{1}{\sqrt{d_k}}                \\
              & \text{\huge0} &   & 1                               \\
              &               &   &   & 1
            \end{array}
            \right),$ в таком случае 
        
            $\tilde{D}^TS^TRS\tilde{D} = \left(
            \begin{array}{ccccc}
                 1                              \\
                           &\ddots &  & \text{\huge0}\\
              &               & 1               \\
              & \text{\huge0} &   & 0            \\
              &               &   &   & 0
            \end{array}
            \right)$. 
    
            Рассмотрим вектор $(S\tilde{D})^T\vec{\xi}$ и его характеристическую функцию. Докажем что он подходит с точностью до линейного преобразования. Действительно, рассмотрим характеристическую функцию этого вектора:
            $\varphi_{(S\tilde{D})^T\vec{\xi}}(\vec{t}) = \varphi_{\vec{\xi}}((S\tilde{D})\vec{t}),$ так как
            \[
            	\varphi_{(S\tilde{D})^T\vec{\xi}}(\vec{t}) = Ee^{i(\vec{t}, (S\tilde{D})^T\vec{\xi})} = \exp(i((S\tilde{D})\vec{t}, \vec{m}) - \frac{1}{2}(R(S\tilde{D})\vec{t}, (S\tilde{D})\vec{t})) = 
            \]
            \[
            	=\exp[i(\vec{t}, (S\tilde{D})^T\vec{m}) - \frac{1}{2}\underbracket{(\tilde{D}^TS^TRS\tilde{D}\vec{t}, \vec{t})}_{=\sum\limits_{i = 1}^{k}t_i^2}] = 
            \]
            \[
            	= \exp[i(\vec{t}, (S\tilde{D})^T\vec{m})]\prod\limits_{i = 1}^k \varphi_{\eta_i}(t_i), 
            \]
            $\eta_i \sim N(0;1)\) и независимы по теореме единственности и теореме независимости в терминах характеристической функции \(\Rightarrow\) вектор \(\vec{\eta} = (S\tilde{D})^T(\vec{\xi} - \vec{m})\)~--- искомый, так как \(\vec{\xi} = ((S\tilde{D})^T)^{-1}\vec{\eta}+\vec{m}.$

            \item Опр 2 $\Rightarrow$ Опр 3. Если $\vec{\xi} = A\vec{\eta} + \vec{b},$ то $(\vec{\lambda}, \vec{\xi}) = (\vec{\lambda}, A \vec{\eta}) + (\vec{\lambda}, \vec{b}) = \underbrace{\lambda^T A\eta}_{\text{сл. вел.}} + \underbrace{\lambda^T b}_{\text{число}}$~--- линейная комбинация независимых нормально распределённых случайных величин. $\Rightarrow$ то есть имеем нормальное распределение.
            
            \item Опр 3 $\Rightarrow$ Опр 1. Пусть $(\xi; \lambda)$~--- нормально распределённая случайная величина, тогда её характеристическая функция $Ee^{i(\xi, \lambda)t} = e^{iE(\xi, \lambda)t - \frac{D(\xi, \lambda)t^2}{2}}.\) Подставим \(t = 1 ~ \Rightarrow Ee^{i(\xi, \lambda)} = e^{i \sum\limits_{k = 1}^{n}\lambda_k E\xi_k - \frac{1}{2}\sum\limits_{k, l = 1}^{n}\lambda_k\lambda_l\operatorname{cov}(\xi_k, \xi_l)} = \exp(i(\vec{\lambda}, E\vec{\xi}) - \frac{1}{2}(R\vec{t}, \vec{t})), ~ R = \operatorname{Var}\vec{\xi}.$
		\end{enumerate}
	\end{proof}

	\subsection{Свойства гауссовских векторов}
	\setcounter{property}{0}
	\begin{property}
		Если \(\vec{\xi} \sim N(\vec{a}, \Sigma),\) то \(\vec{a} = \left(
		\begin{matrix}
		E\xi_1 \\ \vdots \\ E\xi_n
		\end{matrix}
		\right)\)~--- вектор средних, \(\Sigma\)~--- матрица ковариаций.
		\begin{proof}
			Аналогично пункту \(3\) предыдущей теоремы.
		\end{proof}
	\end{property}

	\begin{property}
		Пусть \(\vec{\xi}\sim N(\vec{a}, \Sigma),\) тогда \(\xi_i\) независимы \(\Leftrightarrow~\Sigma\) ~--- диагональна.
		\begin{proof}
			Заметим, что характеристическая функция \(\xi_j\) равна \(\varphi_{\xi_j}(t_j) = e^{it_ja_j - \frac{1}{2}\sigma^2_{jj}t_j^2},\) нужно подставить \(\vec{t} = (0\ldots0, t, 0\ldots0).\) Тогда \((\xi_1, \ldots, \xi_n)\) независимы в совокупности \(\Leftrightarrow ~ \varphi_{\vec{\xi}}(\vec{t}) = \prod\limits_{i =1 }^n \varphi_{\xi_j}(t_j) = e^{i(\vec{a}, \vec{t}) - \frac{1}{2}\sum\limits_{j = 1}^{n}\sigma_{jj}^2t_j^2} \Leftrightarrow \Sigma\) ~--- диагональна.
		\end{proof}
	\end{property}

	\begin{property}[Коши]
		Гауссовские вектора --- нормальные случайные величины.
		\begin{proof}
			Следует из определения \(3\) для \(\lambda_j = (0, \ldots, 0, 1, 0, \ldots, 0)\) (спроецируем гауссовский вектор на одну из компонент).
		\end{proof}
	\end{property}

	\begin{property}
		\(\vec{\xi}\)~--- гауссовский \(\Rightarrow\) любое его линейное преобразование --- гауссовский вектор.
		\begin{proof}
			Пусть \(\vec{\chi} = B\vec{\xi} + \vec{c}.\) По второму определению гауссовского вектора, \(\vec{\chi} = B(A\vec{\eta} + \vec{b}) + \vec{c} = BA\vec{\eta} + B\vec{b} + \vec{c},\) отсюда \(\vec{\chi}\)~--- гауссовский по определению \(2.\)
		\end{proof}
	\end{property}
	
	\begin{property}
		Пусть \(\vec{\xi}\)~--- гауссовский. Тогда его компоненты независимые \(\Leftrightarrow\) они некоррелированы.
		\begin{proof}
			\((\xi_1, \ldots, \xi_n)\)~--- попарно некоррелированы \(\Leftrightarrow \operatorname{cov}(\xi_i, \xi_j) = 0, \forall i \neq j \Leftrightarrow \Sigma\)~--- диагональна \(\Leftrightarrow\) по свойству \(2\) компоненты \(\vec{\xi}\) независимы в совокупности.
		\end{proof}
	\end{property}
	
    \begin{property}[Явный вид плотности многомерного нормального распределения, б/д]
    	$\var \vec{\xi} = \Sigma, \E\vec{\xi} = \vec{m}$. Если матрица ковариации $\Sigma$~--- невырожденная, то $\vec{\xi}$ имеет плотность в $\R^n$.
    	
    	$f(\vec{x}) = \frac{1}{(2\pi)^{n / 2}|\Sigma|^{1/2}} \exp\left(-\frac{1}{2}(\Sigma^{-1}(\vec{x} - \vec{m}), (\vec{x} - \vec{m}))\right), x \in \R^n$.
    \end{property}

\subsection{Многомерная ЦПТ}
    \begin{theorem}[Многомерная ЦПТ]
    	Пусть $\{\vec x_i\}_{i \geqslant 1}$ --- независимые одинаково распределенные случайные вектора, $\E \vec{x}_i = \vec{a}$, $\var \vec{x}_i = \Sigma$.
    	
    	Тогда $\sqrt{n} \left(\frac{\vec{x_1} + \ldots + \vec{x_n}}{n} - \vec{a} \right) \xrightarrow{d} N(\vec{0}, \Sigma), n \rightarrow +\infty$.
    \end{theorem}
    
\begin{note}
	Сходимость векторов по распределению вводится аналогично обычной сходимости случайной величины по распределению, то есть $\forall f: \R^n \rightarrow \R$ непрерывно ограниченных $\E f(\vec{x_n}) \rightarrow \E f(\vec{x})$.
\end{note}