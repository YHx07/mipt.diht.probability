\section{Глава 3.}

\subsection{Интеграл Лебега}
\begin{definition}
    Пусть $A \in \F$. Тогда индикатор множества $A$:
    \begin{equation*}
        I_A(\omega) = 
         \begin{cases}
           1,    &   \omega \in A,   \\
           0,    &   \omega \in \overline{A}.
         \end{cases}
        \end{equation*}
\end{definition}

\begin{definition}
    Случайная величина $\xi$ называется простой, если существует разбиение $\Omega = \sum\limits_{i=1}^nA_i$, такое что $ \xi(\omega) = \sum\limits_{i=1}^n c_iI_{A_i}(\omega)$.
\end{definition}

\begin{definition}
    Пусть $\xi$~--- случайная величина, тогда введем обозначения $\xi_+ = \max(\xi, 0),\  \xi_- = \max(-\xi, 0)$. $\xi = \xi_+ - \xi_-$.
\end{definition}

\begin{lemma}[][б/д]
	$\forall \xi \geqslant 0$ существует набор простых случайных величин $\xi_n$: $\xi_n \uparrow \xi$ ($\xi_n$~--- простая, если $\xi_n = \sum\limits_{i = 1}^k c_iI_{A_i}$).
\end{lemma}

\begin{definition}
	Пусть $\xi$~--- простая случайная величина, то есть $\xi = \sum\limits_{i = 1}^k c_iI_{A_i}$, тогда матожидание $\E \xi = \sum\limits_{i = 1}^k c_i\P(A_i)$, где $\bigsqcup\limits_{i=1}^k A_i = \Omega$.
\end{definition}

\begin{definition}
	Пусть $\xi \geqslant 0$, тогда матожидание $\E \xi = \lim\limits_{n \rightarrow \infty} \E \xi_n$, где $\xi_n \uparrow \xi$, $\xi_n$~--- простые неотрицательные случайные величины, также справедливо равенство $\E \xi = \sup\limits_{\eta \leqslant \xi}\E\eta$, где $\eta$~--- простые неотрицательные случайные величины. 
\end{definition}

\begin{definition}
	Пусть $\xi$~--- произвольные случайные величины. Пусть $\xi_+ = \max(\xi, 0)$, $\xi_- = \max(-\xi, 0) \Rightarrow \xi = \xi_+ - \xi_-$, тогда матожидание 
	$$ \E\xi =
		\begin{tabular}{|c|c|c|} \hline
		$\E\xi_- \setminus \E\xi_+$ & < $+\infty$ & = $+\infty$ \\ \hline
		$< +\infty$ & $\E\xi_+ - \E\xi_-$ & $+\infty$ \\ \hline
		$= +\infty$ & $-\infty$ & $\nexists$ \\ \hline
		\end{tabular}
	$$
\end{definition}
\begin{consequence}
	$\E \xi$~--- конечно $\Leftrightarrow \E|\xi|$~--- конечно.
	\begin{proof}
		$|\xi| = \xi_+ + \xi_-$. $E|\xi|$~--- конечно $\Leftrightarrow \E\xi_+, \E\xi_-$~--- конечны $\Leftrightarrow \E\xi$~--- конечно.
	\end{proof}
\end{consequence}

\begin{statement}
    Таким образом, матожидание случайной величины~--- это интеграл Лебега по мере $\P$, то есть: $$ \E\xi = \int\limits_\Omega \xi dP = \int\limits_\Omega \xi(\omega)P(d\omega).$$

    Для множества $A$:
    $$ \E(\xi\cdot I_A) = \int\limits_A \xi dP. $$
\end{statement}

\noindent {\bfseries Пример:}
    Для случайной величины $\xi \sim \text{Cauchy}(\theta)$ матожидание $\E\xi = +\infty$, то есть матожидание не определено.

\subsection{Свойства матожидания (9 штук)}
\begin{property}
	Пусть $\xi$~--- случайная величина, $\E \xi$~--- конечно, тогда $\forall c \in \R \hookrightarrow \E(c\xi)$~--- конечно и $\E(c\xi) = c\E\xi$.
	\begin{proof}
	    \begin{enumerate}
    	    \item Для простых случайных величин свойство очевидно --- выносим константу $c$ за сумму. 
    		
    		\item Пусть $\xi \geqslant 0$, $\xi_n \uparrow \xi$~--- последовательность простых неотрицательных случайных величин, $c \geqslant 0$. Тогда $c\xi_n \uparrow c\xi \Rightarrow \E(c\xi) = \lim\limits_{n \rightarrow \infty} \E(c\xi_n) = c \lim\limits_{n \rightarrow \infty} \E(\xi_n) = c\E\xi$. 
    		
    		\item В общем случае $\xi = \xi_+ - \xi_-$, тогда $(c\xi)_+ = c\xi_+,\; (c\xi)_- = c\xi_- \Rightarrow \E(c\xi) = \E(c\xi)_+ - \E(c\xi)_- = c\E\xi$. Если $c < 0$, то $(c\xi)_+ = -c\xi_-$ и $(c\xi)_- = -c\xi_+$.
	    \end{enumerate}
	\end{proof}
\end{property}

\begin{property}
	Если $\xi \leqslant \eta, \E \xi, \E \eta$~--- конечны, то $\E \xi \leqslant \E \eta$.
	\begin{proof}
	    \begin{enumerate}
    	    \item Для простых случайных величин~--- очевидно.
    	    
    	    \item Для неотрицательных $\xi$, $\eta$ $\E\xi = \sup\limits_{\mu \leqslant \xi}\E\mu$, где $\mu$~--- простая случайная величина. $\sup\limits_{\mu \leqslant \xi}\E\mu \leqslant \sup\limits_{\mu \leqslant \eta}\E\mu = \E\eta$. 
    	    
    	    \item Пусть $\xi, \eta$~--- произвольные, тогда $\xi_+ \leqslant \eta_+$ и $\xi_- \geqslant \eta_-$. $\E\xi = \E\xi_+ - \E\xi_- \leqslant \E\eta_+ - \E\eta_- = \E\eta$.
	    \end{enumerate}
	\end{proof}
\end{property}
\begin{property}
	Если $\E\xi$~--- конечно, то $|\E\xi| \leqslant \E|\xi|$.
	\begin{proof}
		$|\xi| = \xi_+ + \xi_- \Rightarrow \E|\xi|$~--- конечно. $\underbrace{-\E\xi_+ - \E\xi_-}_{-\E|\xi|} \leqslant \underbrace{\E\xi_+ - \E\xi_-}_{\E\xi} \leqslant \underbrace{\E\xi_+ + \E\xi_-}_{\E|\xi|}$.
	\end{proof}
\end{property}

\begin{property}[Аддитивность]
	Пусть $\xi$ и $\eta$~--- случайные величины, $\E\xi$ и $\E\eta$~--- конечные, тогда $\E(\xi + \eta) = \E\xi + \E\eta$.
	\begin{proof}
	    \begin{enumerate}
    	    \item Для простых случайных величин~--- очевидно. 
    	    
    	    \item Пусть $\xi, \eta \geqslant 0$, возьмем $\xi_n \uparrow \xi$, $\eta_n \uparrow \eta$~--- простые и положительные. Тогда $\xi_n + \eta_n \uparrow \xi + \eta \Rightarrow \E(\xi + \eta) = \lim\limits_{n \rightarrow \infty}\E(\xi_n + \eta_n) = \lim\limits_{n \rightarrow \infty}\E \xi_n + \lim\limits_{n \rightarrow \infty} \E\eta_n = \E\xi + \E\eta$. 
    	    
    	    \item Пусть $\xi, \eta$~--- произвольные, тогда $(\xi + \eta)_+ \leqslant \xi_+ + \eta_+$. Пусть $\delta = (\xi_+ + \eta_+) - (\xi + \eta)_+ \Rightarrow \E\delta + \E(\xi + \eta)_+ = \E\xi_+ + \E\eta_+ \Rightarrow \E(\xi + \eta)_+ = \E\xi_+ + \E\eta_+ - \E\delta$. Аналогично, $\E(\xi + \eta)_- = \E\xi_- + \E\eta_- - \E\delta$. Тогда $\E(\xi + \eta) = \E(\xi + \eta)_+ - \E(\xi + \eta)_- = \E\xi_+ + \E\eta_+ - \E\delta - \E\xi_- - \E\eta_- + \E \delta = \E\xi + \E\eta$. Рассмотрим $(\xi + \eta)_- = (\xi + \eta)_+ - (\xi + \eta) = \xi_+ + \eta_+ - \delta - (\xi + \eta) = \xi_- + \eta_- - \delta$.
	    \end{enumerate}
	\end{proof}
\end{property}

\begin{property}
	\begin{enumerate}
		\item Пусть $|\xi| \leqslant \eta$, $\E\eta$~--- конечное, тогда $\E \xi$~--- конечная.
		\item Пусть $\xi \leqslant \eta$, $\E\eta$~--- конечное, тогда $\E\xi < + \infty$. \\
		Пусть $\xi \geqslant \eta$, $\E\eta$~--- конечное, тогда $\E\xi > - \infty$.
		\item Если $\E\xi$~--- конечное и $A \in \F$, то $\E(\xi\cdot I_A)$~--- конечное.
	\end{enumerate}
	\begin{proof}
		\begin{enumerate}
			\item $\xi_-, \xi_+ \leqslant \eta \Rightarrow 0 \leqslant \E\xi_+ = \sup\limits_{0 \leqslant \mu \leqslant \xi_+}\E\mu \leqslant \E\eta < +\infty \Rightarrow \E\xi_+, \E\xi_- < +\infty \Rightarrow \E\xi$~--- конечное.
			\item $\xi_+ \leqslant \eta_+ < +\infty \Rightarrow$ по первому пункту $\E\xi_+ < +\infty \Rightarrow \E\xi < +\infty$.
			\item $(\xi\cdot I_A)_+ = I_A\cdot\xi_+ < \xi_+ \Rightarrow \E(\xi\cdot I_A)_+$~--- конечное. Аналогично $\E(\xi\cdot I_A)_-$~--- конечное $\Rightarrow \E(\xi\cdot I_A)$~--- конечное.
		\end{enumerate}
	\end{proof}
\end{property}

\begin{definition}
	Событие $A$ происходит почти наверное, если $\P(A) = 1$.
\end{definition}

\begin{property}
	Если $\xi = 0$ почти наверное, то $\E\xi = 0$.
	\begin{proof}
	    \begin{enumerate}
	    \item Пусть $\xi$~--- простая случайная величина, то есть $\xi = \sum\limits_{k = 1}^nx_kI_{A_k}$, где $\{x_k\}$ различные, $\{A_k\}$~--- разбиение $\Omega$, $A_k = \{\xi = x_k\}$. Тогда если $x_k \neq 0$, то $A_k = \{\xi = x_k\} \subseteq \{\xi \neq 0\} \Rightarrow \P(A_k) \leqslant \P(\xi \neq 0) = 0 \Rightarrow \E\xi = \sum\limits_{k = 1}^n x_k\P(A_k) = 0$. 
	    
	    \item Если $\xi \geqslant 0$, то $\E\xi = \sup\limits_{\xi \geqslant \eta} \E\eta$, где $\eta$~--- простые $\Rightarrow \E\xi \geqslant 0$. Но $0 \leqslant \eta \leqslant \xi = 0$ почти наверное $\Rightarrow \E\eta = 0 \Rightarrow \E\xi = 0$. 
	    
	    \item Пусть $\xi$~--- произвольные $\Rightarrow \xi_+ = 0$ почти наверное, $\xi_- = 0$ почти наверное и $\E\xi = \E\xi_+ - \E\xi_- = 0.$
	    \end{enumerate}
	\end{proof} 
\end{property}


\begin{consequence} (Абсолютная непрерывность интеграла Лебега.)
    $$ \P(A) = 0, A \in \F \Rightarrow \int\limits_{A} \xi d\P = 0. $$
    \begin{proof}
        $\xi \cdot I_A = 0$ п.н. $\Rightarrow 0 = \E(\xi \cdot I_A) = \int\limits_A\xi d\P$.
    \end{proof}
\end{consequence}

\begin{property}
	Если $\xi = \eta$ почти наверное и $\E | \eta| < +\infty$, то $\E | \xi| < +\infty$ и $\E \xi = \E \eta$.
	\begin{proof}
		Пусть $A = \{ \xi \neq \eta \}$, тогда $I_A = 0$ почти наверное, следовательно $\xi \cdot I_A = 0$ почти наверное и $\eta \cdot I_A = 0$ почти наверное. Так как $\xi = \xi \cdot I_A + \xi \cdot I_{\overline A}$, то $\xi = \xi \cdot I_A + \eta \cdot I_{\overline A}$, потому что на $\overline A$ выполняется $\xi = \eta$. Из свойства 6 имеем $\E \xi = \E ( \xi \cdot I_A ) + \E (\eta \cdot I_{\overline A} ) = \E ( \eta \cdot I_A) + E ( \eta \cdot I_{\overline A} ) = \E \eta$. 
	\end{proof}
\end{property}

\begin{property}
	Пусть $\xi \geqslant 0$ и $\E \xi = 0$, тогда $\xi = 0$ почти наверное.
	\begin{proof}
		Рассмотрим события $A = \{ \xi > 0 \}$ и $A_n = \left\{ \xi > \frac{1}{n} \right\} $, следовательно, $A_n \uparrow A$. Имеем $\P (A_n) = \E I_{A_n}$, так как $n\xi > 1$ на $A_n$, то  $\E I_{A_n} \leqslant \E (n\xi \cdot I_A) \leqslant n \E \xi = 0$, значит, $\P(A) = \lim\limits_{n \rightarrow +\infty} \P (A_n) = 0$.
	\end{proof}
\end{property}

\begin{property}
	Пусть $\E \xi$ и $\E \eta$ конечны, $\forall A \in \F \hookrightarrow \E (\xi \cdot I_A) \leqslant \E ( \eta \cdot I_A)$. Тогда  $\xi \leqslant \eta$ почти наверное.
	\begin{proof}
		Рассмотрим событие $B = \{ \xi > \eta \}$. Из условия и построения $B$ получаем, что $\E (\eta \cdot I_B) \leqslant \E (\xi \cdot I_B) \leqslant \E (\eta \cdot I_B)$, следовательно, $\E( \xi \cdot I_B) = \E(\eta \cdot I_B)$, значит $\E \big( (\xi - \eta) \cdot I_B \big) = 0$. Так как $(\xi - \eta) \cdot I_B \geqslant 0$, то по свойству 8 $(\xi - \eta) \cdot I_B = 0$ почти наверное, следовательно $I_B = 0$ почти наверное, потому что $\xi- \eta > 0$ на $B$.
	\end{proof}
\end{property}

\begin{theorem}[о математическом ожидании произведения случайных величин]
	Пусть $\xi \indep \eta$, причем $\E \xi$ и $\E \eta$ конечны, тогда $\E \xi \eta$ конечно и $\E \xi \eta = \E \xi \cdot \E \eta$.
	\begin{proof}
	    \begin{enumerate}
	        \item Пусть $\xi$ и $\eta$~--- простые случайные величины, то есть $\xi$ принимает значения $\{ x_1, \ldots, x_n \}$, $\eta$ принимает значения $\{ y_1, \ldots, y_n \}$. Тогда по линейности
    		\begin{multline*}
    			\E \xi \eta = \sum\limits_{k,j=1}^{n} x_k y_j \P (\xi = x_k, \eta = y_j) = \sum\limits_{k,j=1}^n x_k y_j \P(\xi = x_k) \cdot \P(\eta = y_j) = \\ = \sum\limits_{k = 1}^n x_k \P (\xi = x_k) \sum\limits_{j = 1}^{n} y_j \P (\eta = y_j) = \E \xi \cdot \E \eta.
    		\end{multline*}
    			
    		\item Рассмотрим $\xi_n \uparrow \xi$, 
    		$$\xi_n = \sum\limits_{k=0}^{n \cdot 2^n - 1} \dfrac{k}{2^n} I \left(\dfrac{k}{2^n} \leqslant \xi \leqslant \dfrac{k+1}{2^n} \right) + n I(\xi > n),$$
    		 следовательно, $\xi_n = \varphi_n(\xi)$, значит, $\xi_n$~--- $\F_\xi$-измеримая. 
    		 
    		 Пусть $\xi, \eta \geqslant 0$. Существует последовательность $\F_\xi$-измеримых ($\F_\eta$-измеримых) простых неотрицательных простых функций $\xi_n \uparrow \xi$ ($\eta_n \uparrow \eta$). Так как $\xi \indep \eta$, то $\xi_n = \varphi_n (\xi) \indep \varphi_n(\eta) = \eta_n$. Следовательно, $\xi_n \cdot \eta_n \uparrow \xi \cdot \eta$, а по определению математического ожидания $\E \xi\eta = \lim\limits_{n \rightarrow +\infty} \E (\xi_n \eta_n) =  \lim\limits_{n \rightarrow +\infty} \E \xi_n \cdot \E \eta_n = \E \xi \cdot \E \eta$.
    			
    		\item Пусть теперь $\xi$ и $\eta$~--- произвольные случайные величины. $\xi^+$ и $\xi^-$~--- функции от $\xi$, $\eta^+$ и $\eta^-$~--- функции от $\eta$, следовательно, $\xi^+ \indep \eta^+$ и $\xi^- \indep \eta^-$, отсюда $(\xi \eta)^+ = \xi^+ \eta^+ + \xi^- \eta^-$ значит, $\E (\xi \eta)^+ = \E \xi^+ \eta^+ + \E \xi^- \eta^- = \E \xi^+ \E \eta^+ + \E \xi^- \E \eta^-$, аналогично $\E (\xi \eta)^- = \E \xi^+ \eta^- + \E \xi^- \eta^+ = \E \xi^+ \E \eta^- + \E \xi^- \E \eta^+$. Осталось заметить, что $\E \xi \eta = \E (\xi \eta)^+ - \E ( \xi \eta)^- = \E \xi^+ \E \eta^+ + \E \xi^- \E \eta^- - \E \xi^+ \E \eta^- - \E \xi^- \E \eta^+ =  (\E \xi^+ - \E \xi^-) (\E \eta^+ - \E \eta^-) = \E \xi \cdot \E \eta$.
	    \end{enumerate}
	\end{proof}
\end{theorem}
	
Пусть $\xi = \sum\limits_{i = 1}^n x_i \cdot I (\xi = x_i)$~--- простая случайная величина. Тогда $\E g(\xi) = \sum\limits_{i=1}^{n} g(x_i) \cdot \P(\xi = x_i) = \sum\limits_{i = 1}^n g(x_i) \Delta F_\xi (x_i)$, где $\Delta F_\xi(x_i) = F_\xi(x_i) - F_\xi (x_i - 0)$.

\begin{theorem}[о замене переменной в интеграле Лебега][б/д]
	Пусть 
	\begin{enumerate}
	    \item $(\Omega, \F)$ и $(E, \mathcal{E})$~--- два измеримых пространства;
	    \item $X : \Omega \rightarrow E$~--- $\F$-измеримая функция, то есть $\forall B \in \mathcal{E} \hookrightarrow X^{-1}(B) \in \F$;
	    \item $\P$~--- вероятностная мера на $(\Omega, \F)$;
	    \item $\P_X$~--- вероятностная мера на $(E, \mathcal{E})$, заданная по правилу $\P_X(A) = \P (\omega: X(\omega) \in A)$ для $A \in \mathcal{E}$.
	\end{enumerate}
	
	Тогда для любой $\mathcal{E}$-измеримой функции $g(x):E \rightarrow \R$, то есть $\forall B \in \B(\R) \hookrightarrow g^{-1}(B) \in \mathcal{E}$, верно, 
	$$\int\limits_A g(x) \P_X(dx) = \int\limits_{X^{-1}(A)} g\big( X(\omega) \big) \P(d \omega).$$
\end{theorem}

\begin{consequence}[1]
    Пусть $\xi : \Omega \rightarrow \R (\R^n)$, в таком случае вероятностная мера $\P_\xi$ однозначно восстанавливается по $F_\xi$, следовательно, по теореме $\E g(\xi) = \int g(\xi) \, d\P = \int g(x) \P_\xi(dx) = \int g(x) \, d F_\xi (x)$.
\end{consequence}
\begin{consequence}[2]
    Пусть  $\xi$~--- абсолютно непрерывная случайная величина с плотностью $p_\xi(x)$, тогда $d F_\xi (x) = p_\xi (x) \, dx$, следовательно $\E g(x) = \int\limits_{\R} g(x) p_\xi(x) \, d x$.
\end{consequence}

\subsection{Прямое произведение вероятностных пространств и формула свертки}

\begin{definition}
	Пусть $(\Omega_1, \F_1, \P_1)$ и $(\Omega_2, \F_2, \P_2)$~--- два вероятностных пространства. Тогда $(\Omega, \F, \P)$~--- их прямое произведение, если 
	\begin{enumerate}
		\item $\Omega = \Omega_1 \times \Omega_2$, то есть $\omega = (\omega_1, \omega_2)$;
		\item $\F = \F_1 \otimes \F_2$, то есть $\F = \sigma\big\{ \{ B_1 \times B_2 \} | B_1 \in \F_1, B_2 \in \F_2 \big\}$;
		\item $\P = \P_1 \otimes \P_2$, то есть  $\P$~--- продолжение вероятностной меры $\P_1 \times \P_2$, заданное на прямоугольнике $B_1 \times B_2, B_1 \in \F_1, B_2 \in \F_2$ по правилу $\P(B_1 \times B_2) = \P_1 (B_1) \cdot \P_2 (B_2)$. Так как $\{B_1 \times B_2 \}$~--- полукольцо, то $\P$ существует и единственна по теореме Каратеодори.
	\end{enumerate}
\end{definition}
\begin{theorem}[Фубини][б/д]
	Пусть 
	\begin{enumerate}
	    \item $(\Omega, \F, \P)$~--- прямое произведение вероятностных пространств $(\Omega_1, \F_1, \P_1)$ и $(\Omega_2, \F_2, \P_2)$.
	    \item $\xi : \Omega \rightarrow \R$ такая, что $\int\limits_\Omega \big| \xi (\omega_1, \omega_2) \big| \, d \P < + \infty$.
	\end{enumerate}
     Тогда интегралы $$\int\limits_{\Omega_1} \xi (\omega_1, \omega_2) \P_1 (d \omega_1) \text{ и } \int\limits_{\Omega_2} \xi(\omega_1, \omega_2) \P_2 (d \omega_2)$$
     \begin{enumerate}
         \item определены почти наверное относительно $\P_2$ и $\P_1$ соответственно;
         \item являются измеримыми случайными величинами относительно $\F_2$ и $\F_1$.
         \item $$\int\limits_\Omega \xi (\omega_1, \omega_2) \, d \P = \int\limits_{\Omega_2} \int\limits_{\Omega_1} \xi (\omega_1, \omega_2) \P_1 (d \omega_1) \P_2 (d \omega_2) = \int\limits_{\Omega_1} \int\limits_{\Omega_2} \xi(\omega_1, \omega_2) \P_2 (d \omega_2) \P_1 (d \omega_1).$$
     \end{enumerate}
	 Из всего этого следует, что двойной интеграл равен повторному.
\end{theorem}

\begin{statement}
 	Пусть $\xi \indep \eta$~--- случайные величины, тогда $(\R^2, \B(\R^2), \P_{(\xi, \eta)}) = (\R, \B(\R), \P_\xi) \otimes (\R, \B(\R), \P_\eta)$.
 	\begin{proof} Достаточно проверить свойство прямого произведения:
 		\begin{enumerate}
 			\item $\R^2 = \R \times \R$;
 			\item $\B(\R^2) = \sigma(\B(\R) \times \B(\R))$ по определению борелевской $\sigma$-алгебры в $\R^2$;
 			\item $\P_{(\xi, \eta)} (B_1 \times B_2) = \P(\xi \in B_1, \eta \in B_2) = \P(\xi \in B_1) \cdot \P(\eta \in B_2) = \P_\xi(B_1) \cdot \P_\eta (B_2)$. \qedhere
 		\end{enumerate}
 	\end{proof}
\end{statement}

\begin{lemma}[о свертке]
	Пусть случайные величины $\xi$ и $\eta$ независимы c функциями распределения $F_\xi$ и $F_\eta$. Тогда
	\begin{enumerate}
	    \item Выполняется равенство: $$ F_{\xi + \eta} (z) = \int\limits_\R F_\xi (z - x) \, dF_\eta (x) = \int\limits_\R F_\eta(z - x) \, d F_\xi(x).$$
	    \item Если $\xi$ и $\eta$ имеют плотности распределения $f_\xi$ и $f_\eta$ соответственно, то $\xi + \eta$ имеет плотность распределения $$f_{\xi + \eta} (z) = \int\limits_\R f_\xi (z - x) f_\eta (x) \, dx = \int\limits_\R f_\eta (z - x) f_\xi(x) \, dx$$.
	\end{enumerate} 
	
	\begin{proof}
		Заметим, $F_{\xi + \eta} (z) = \P ( \xi + \eta \leqslant z) = \E I(\xi + \eta \leqslant z)$, а по теореме о замене переменных в интеграле Лебега это равно $\int\limits_{\R^2} I(x + y \leqslant z) \P_\xi(dx) \P_\eta(dy)$, полученный двойной интеграл по Фубини можно записать как повторный:
		$$ \int\limits_\R \left( \int\limits_\R I(x + y \leqslant z) \P_\xi(dx) \right) \P_\eta(dy) = \int\limits_\R \left( \int\limits_{-\infty}^{z-y} \P_\xi(dx) \right) \P_\eta(dy) = \int\limits_\R F_\xi(z - y) \, dF_\eta(y).$$
		Перейдем ко второму пункту доказательства:
		\begin{multline*}
			F_{\xi + \eta} (z) = \int\limits_{\R^2} I(x + y \leqslant z) \P_\xi(dx) \P_\eta(dy) = \int\limits_{\R^2} I(x + y \leqslant z) f_\xi(x) f_\eta(y) \, dx\,dy \overset{t = x+y}{=} \\ \overset{t = x+y}{=} \int\limits_{\R^2} I(t \leqslant z) f_\xi(x) f_\eta(t - x) \, dx\,dt = \int\limits_{-\infty}^z \left( \int\limits_\R f_\xi(x) f_\eta(t-x) \, dx \right) \, dt.
		\end{multline*}
		Следовательно, по определению плотности, $f_{\xi+\eta}(t) = \int\limits_\R f_\xi(x) f_\eta(t-x) \, dx$.
	\end{proof}
\end{lemma}

\subsection{Дисперсия и ковариация}
\begin{definition}
	Дисперсией случайной величины $\xi$ называется $\D \xi = \E (\xi - \E \xi)^2 $, если $\E{\xi} < + \infty$. Очевидно, $\D{\xi} \geqslant 0$.
\end{definition}

\begin{definition}
	Ковариацией двух случайных величин называется $\cov (\xi, \eta) = \E \big( ( \xi - \E \xi)(\eta - \E \eta ) \big)$. Легко заметить, что $\cov (\xi, \xi ) = \D \xi$.
\end{definition}

\begin{definition}
    Если $\cov(\xi, \eta) = 0$, то случайные величины $\xi$ и $\eta$ называются некоррелированными.
\end{definition}

\begin{definition}
	Величина $\rho ( \xi, \eta ) = \dfrac{\cov ( \xi, \eta )}{\sqrt{\D \xi \cdot \D \eta}}$ называется коэффициентом корреляции случайных величин $\xi$ и $\eta$ при условии, что $\D \xi$ и $\D \eta$ не равны нулю и конечны.
\end{definition}

\subsection{Свойства ковариации и дисперсии (7 штук)}
\setcounter{property}{0}
\begin{property}
	$\cov( a \xi + b \zeta, \eta ) = a \cov (\xi, \eta) + b \cov (\zeta, \eta)$. Ковариация билинейна.
\end{property}

\begin{property}
	$\cov( \xi,  \eta ) = \E \xi \eta - \E \xi \cdot \E \eta ~\Rightarrow ~ \D \xi = \E \xi^2 - ( \E \xi )^2$.
	\begin{proof}
		$\cov (\xi, \eta ) = \E (\xi - \E \xi)( \eta - \E \eta) = \E \xi \eta - \E \big( ( \E \xi ) \cdot \eta \big) - \E \big( (\E \eta ) \cdot \xi \big) + \E \xi \cdot \E \eta = \E \xi \eta - \E \xi \cdot \E \eta $
	\end{proof}
\end{property}

\begin{property}
	Пусть $c \in \R$, тогда $\D (c \xi ) = c^2 \D \xi$, $\D (\xi + c ) =  \D \xi$, $\D c = 0$.
	\begin{proof}
		\vspace{-2pc}
		\begin{align*}
			\D (c \xi ) &= \E c^2 \xi^2 - \left( \E c \xi \right)^2 = c^2 \E \xi^2 - c^2 \left( \E \xi \right)^2 = c^2 \D \xi;\\
			\D ( c + \xi ) &= \E \big( c + \xi - \E ( c + \xi ) \big)^2 = \E \left( c + \xi - c - \E \xi  \right)^2 = \D \xi; \\
			\D c &= \E (c - \E c )^2 = \E (c - c)^2 = 0. \qedhere
		\end{align*} 
	\end{proof}
\end{property}

\begin{property}[Неравенство Коши-Буняковского]
	$\left| \E \xi \eta \right|^2 \leqslant \E \xi^2 \cdot \E \eta^2$
	\begin{proof}
		Рассмотрим для $\lambda \in \R$ функцию $f( \lambda ) = \E (\xi - \lambda \eta)^2 \geqslant 0$. Имеем $f( \lambda ) = \E \xi^2 + 2 \lambda \E \xi \eta + \lambda^2 \E \eta^2 \geqslant 0$. Для выполнения неравенства дискриминант  полученного многочлена должен быть меньше нуля: $D = 4 \E \xi \eta - 4 \E \xi^2 \eta^2 \leqslant 0$, откуда следует неравенство.
	\end{proof}
\end{property}

\begin{property}
	$| \rho (\xi , \eta ) | \leqslant 1$, причем $\rho ( \xi , \eta ) = \pm 1 ~\Leftrightarrow~ \xi = a \eta + b$ почти наверное.
	\begin{proof}
	    \begin{enumerate}
    		\item Рассмотрим случайные величины $\xi_1 = \xi - \E \xi$ и $\eta_1 = \eta - \E \eta$, следовательно $\rho ( \xi_1 , \eta_1 ) = \dfrac{\E \xi_1 \eta_1}{\sqrt{\E \xi_1^2 \cdot \E \eta_1^2}} \leqslant 1$ по неравенству Коши-Буняковского. 
    		
    		\item Пусть $\rho (\xi_1, \eta_1) = 1$. Из $\dfrac{\E \xi_1 \eta_1}{\sqrt{\E \xi_1^2 \cdot \E \eta_1^2}} = 1$ получим: $(\E \xi_1 \eta_1)^2 = \E \xi_1^2 \cdot \E \eta_1^2$.
    		
    		Рассмотрим для $\lambda \in \R$ функцию $f( \lambda ) = \E (\xi_1 - \lambda \eta_1)^2 = \E \xi^2 + 2 \lambda \E \xi \eta + \lambda^2 \E \eta^2 \geqslant 0$, учитывая полученное ранее: $\dfrac{D}{4} = (\E\xi_1\eta_1)^2 - \E\xi_1^2\E\eta_1^2 = 0$, следовательно, $\exists !\lambda_0: f(\lambda_0) = 0$,  то есть $\E ( \xi_1 + \lambda_0 \eta_1 )^2 = 0$, отсюда $(\xi_1 + \lambda_0 \eta )^2 = 0$ почти наверное, а, значит, и $\xi_1 + \lambda_0 \eta  = 0$ почти наверное. Теперь можно заключить, что $\xi = \E \xi - \lambda_0 (\eta - \E \eta )$.
		\end{enumerate}
	\end{proof}
\end{property}

\begin{property} 
	Если $\xi \indep \eta$, то $\cov (\xi, \eta) = 0$, обратное неверное.
	\begin{proof}
			$\cov( \xi, \eta) = \E \xi \eta - \E \xi \cdot \E \eta$, но так как $\xi \indep \eta$, то $\E \xi \eta = \E \xi \cdot \E \eta$, следовательно, $\cov( \xi, \eta ) = 0$.
	\end{proof}
\end{property}

\begin{lemma}
	Пусть $\xi_1, \ldots, \xi_n$~--- попарно некоррелированные случайные величины (например, независимые в совокупности), $\D \xi_1 + \ldots + \D \xi_n < +\infty$, тогда $\D ( \xi_1 + \ldots + \xi_n ) = \D \xi_1 + \ldots + \D \xi_n$.
	\begin{proof}
		$$\D \left( \sum_{i=1}^n \xi_i \right) = \cov \left( \sum_{i=1}^n \xi_i, \sum_{j=1}^n \xi_j \right) = \sum_{i, j = 1}^n \cov (\xi_i, \xi_j).$$
		По условию, если $i \neq j$, то $\cov (\xi_i, \xi_j) = 0$, следовательно 
		$$\D \left( \sum_{i=1}^n \xi_i \right) = \sum_{i=1}^n \cov (\xi_i, \xi_i) = \sum_{i=1}^n \D \xi_i.$$
	\end{proof}
\end{lemma}

\subsection{Многомерный случай}
\begin{definition}
	Пусть $\vec \xi  = (\xi_1, \ldots, \xi_n )$~--- случайный вектор, тогда его математическим ожиданием называется вектор из математических ожиданий его компонент, то есть $\E \vec \xi = (\E \xi_1, \ldots, \E \xi_n)$.
\end{definition}
\begin{definition}
	Матрицей ковариаций случайного вектора $\vec \xi$ называется 
	$$ \var \vec \xi = \begin{pmatrix}
		\cov ( \xi_1, \xi_1) & \cdots & \cov (\xi_1, \xi_n)\\
		\vdots  & \ddots & \vdots\\
		\cov ( \xi_n, \xi_1) & \cdots & \cov (\xi_n, \xi_n)
	\end{pmatrix} = \left\| \cov (\xi_i, \xi_j ) \right\|_{i, j = 1}^n.$$
\end{definition}

\begin{lemma}
	Матрица ковариаций случайного вектора~--- симметрическая и неотрицательно определенная\footnote{Матрица $A$ неотрицательно определена, если $\forall \vec x \in \R^n: {\vec x}^T A \vec x \geqslant 0$}.
	\begin{proof}
		Матрица $\var \vec \xi = \left\| \cov (\xi_i, \xi_j ) \right\|_{i, j = 1}^n$~--- симметрическая, так как $r_{ij} \equiv \cov ( \xi_i, \xi_j ) = \cov (\xi_j, \xi_i) \equiv r_{ji}$. Пусть $\vec x \in \R^n$, тогда 
		\begin{multline*}
			{\vec x}^T \var \vec \xi \vec x = ( \vec x, \var \vec \xi \vec x ) = \sum_{i,j=1}^n \cov (\xi_i, \xi_j ) x_i x_j = \sum_{i,j=1}^n \cov (x_i \xi_i, x_j \xi_j) =\\ = \cov \left( \sum_{i=1}^n  x_i \xi_i, \sum_{j=1}^n x_j \xi_j \right) = \cov \left( \sum_{i=1}^n  x_i \xi_i, \sum_{i=1}^n x_i \xi_i \right) =\D \left( \sum_{i=1}^n x_i \xi_i \right) \geqslant 0.
		\end{multline*}
	\end{proof}
\end{lemma}

\subsection{Неравенства (3 штуки)}
\begin{lemma}[Неравенство Маркова]
	Пусть $\xi \geqslant 0$~--- случайная величина, $\E \xi < +\infty$. Тогда $\forall \varepsilon > 0 \hookrightarrow \P (\xi \geqslant \varepsilon ) \leqslant \dfrac{\E \xi}{\varepsilon}$.
	\begin{proof}
		$\P (\xi \geqslant \varepsilon ) = \E I(\xi \geqslant \varepsilon)$. На множестве ${ \xi \geqslant \varepsilon}$ случайная величина $\dfrac{\xi}{\varepsilon} \geqslant 1$, следовательно $\E I(\xi \geqslant \varepsilon) \leqslant \E \left( \dfrac{\xi}{\varepsilon} \cdot I (\xi \geqslant \varepsilon) \right) \leqslant \dfrac{1}{\varepsilon} \cdot  \E \xi$.
	\end{proof}
\end{lemma}

\begin{lemma}[Неравенство Чебышёва]
	Пусть $\xi$~--- случайная величина такая, что $\D \xi < + \infty$, тогда $\forall \varepsilon > 0 \hookrightarrow \P \big(|\xi - \E \xi | \geqslant \varepsilon \big) \leqslant \dfrac{\D \xi}{\varepsilon^2}$.
	\begin{proof}
		$\P \big(|\xi - \E \xi | \geqslant \varepsilon\big) = \P \big( | \xi - \E \xi |^2 \geqslant \varepsilon^2 \big)$. Из неравенства Маркова имеем, что $\P \big( | \xi - \E \xi |^2 \geqslant \varepsilon^2 \big) \leqslant \dfrac{\E (\xi - \E \xi )^2}{\varepsilon^2} = \dfrac{\D \xi}{\varepsilon^2}$.
	\end{proof} 
\end{lemma}

\begin{lemma}[Неравенство Йенсена]
	Пусть $g(x)$~--- борелевская выпуклая вниз (вверх) функция и $\E \xi < +\infty$. 
	Тогда $\E g(\xi) \geqslant g(\E \xi )$ (~$\E g(\xi) \leqslant g(\E \xi )$\,).
	\begin{proof}
		Так как $g(x)$ выпукла вниз, то $\forall x_0 \in \R : g(x) \geqslant g(x_0) + \lambda(x_0)(x - x_0)$. Положим $x = \xi$ и $x_0 = \E \xi$, тогда $g(\xi) \geqslant g(\E \xi) + \lambda(\E \xi)(\xi - \E \xi)$, считая математическое ожидание от обоих частей неравенства, получаем $\E g(\xi) \geqslant g( \E \xi ) + 0$.
	\end{proof}
\end{lemma}

%\begin{definition}
%	Пусть $\xi$ и $\{\xi_i\}_{i=1}^{+\infty}$~--- случайные величины, тогда $\xi_n \overset{\P}{\rightarrow} \xi$ сходится по вероятности, если $\forall \varepsilon > 0 : \P \big(\omega : | \xi_n (\omega) - \xi ( \omega ) | > \varepsilon \big) \rightarrow 0$ при $n \rightarrow +\infty$.
%\end{definition}

\begin{theorem}[ЗБЧ в форме Чебышёва]
	Пусть 
	\begin{enumerate}
	    \item $\{\xi_i\}_{i=1}^{+\infty}$~---  попарно некоррелированные случайные величины, причем $\forall n \hookrightarrow \D \xi_n \leqslant C$.
	    \item Обозначим $S_n = \sum\limits_{i=1}^n \xi_i$.
	\end{enumerate} 
	Тогда $\forall \varepsilon > 0 \hookrightarrow \P \left(\left| \dfrac{S_n - \E S_n}{n} \right| > \varepsilon\right) \rightarrow 0$ при $n \rightarrow +\infty$.
	
	То же самое: $\dfrac{S_n - \E S_n}{n} \overset{\P}{\rightarrow} 0$ при $n \rightarrow +\infty$\Big.
	\begin{proof}
		По неравенству Чебышёва 
		$$\P \left(\left| \dfrac{S_n - \E S_n}{n} \right| > \varepsilon\right) \leqslant \dfrac{\D S_n}{n^2 \varepsilon^2} \leqslant \dfrac{nC}{n^2\varepsilon^2} \rightarrow 0.$$ 
	\end{proof}
\end{theorem}

\begin{consequence}
	Пусть $\{ \xi_n \}_{i=1}^{+\infty}$~--- независимые случайные величины такие, что:
	\begin{enumerate}
	    \item $\forall n \in \N \hookrightarrow \D \xi_n \leqslant C$,
	    \item $\E \xi_n = a$.
	\end{enumerate} 
	 Тогда $\forall \varepsilon > 0 \hookrightarrow \P \left(\left| \dfrac{S_n}{n} - a \right| > \varepsilon\right) \rightarrow 0$ при $n \rightarrow +\infty$.	
		
	То же самое: $\dfrac{S_n}{n} \xrightarrow{\P} a$ при $n \rightarrow + \infty$. 
\end{consequence}