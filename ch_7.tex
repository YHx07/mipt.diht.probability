\section{Глава 7.}

\subsection{Характеристические функции}
\begin{definition}
	Характеристической функцией случайной величины $\xi$ называется $\varphi_\xi(t) = \E e^{it\xi}, t \in \R$. (Формально это преобразование Фурье случайной величины $\xi$).
\end{definition}
\begin{definition}
	Пусть $F(x)$~--- функция распределения, тогда ее характеристическая функция $\varphi_F (t) = \int\limits_\R e^{i t x} \, d F(x)$.
\end{definition}
 Если $F_\xi(x)$~--- функция распределения случайной величины $\xi$, то характеристические функции $\xi$ и $F_\xi$ совпадают.\\
 
Чтобы считать интеграл Лебега от комплекснозначной функции, будем пользоваться формулой Эйлера: $\varphi_\xi(t) = \E e^{i t \xi} = \E \cos (t \xi) + i \E \sin(t \xi)$.
 
 \begin{definition}
 	Пусть $\vec \xi = ( \xi_1, \ldots, \xi_n)$~--- случайный вектор. Его характеристической функцией называется $\varphi_{\vec \xi} \left(\vec t \right) = \E e^{i \left( \vec t, \vec \xi \right)}, t \in \R^n$.
 \end{definition}
 \begin{definition}
 	Пусть $F \left( \vec x \right), \vec x \in \R^n$~--- функция распределения в $\R^n$, тогда его характеристической функцией называется $\varphi_F \left( \vec t \right) = \int\limits_\R e^{i \left( \vec t, \vec x \right)} \, d F \left( \vec x \right), \vec x \in \R^n$.
 \end{definition}
 \subsection{Свойства характеристических функций}
 Свойство №0. Харфункция существует всегда.
 \setcounter{property}{0}
 \begin{property}
 	Пусть $\varphi(t)$~--- характеристическая функция случайной величины $\xi$, тогда $| \varphi (t) | \leqslant \varphi(0) = 1$.
 	\begin{proof}
 		$| \varphi(t) | = | \E e^{i t \xi} | \leqslant \E | \underbracket[0.5pt]{e^{i t \xi}}_{\equiv 1} | = 1 = \varphi(0)$. ($\cos t\xi + i \sin t\xi$, модуль этого комплексного числа равен единице).
 	\end{proof}
 \end{property}
 
 \begin{property}
		Пусть $\varphi(t)$~---характеристическая функция случайной величины $\xi$, а $\eta = a\xi + b$, где $a, b \in \mathbb{R}$, тогда $\varphi_\eta(t) = e^{itb}\cdot \varphi_\xi(at).$
		\begin{proof}
			$\varphi_\eta(t) = \E e^{it\eta} = \E e^{it(a\xi + b)} = e^{itb} \E e^{ita\xi} = e^{itb}\cdot \varphi_\xi(at)$ ($e^{itb} = const$, значит можем вытащить за матожидание).
		\end{proof}
	\end{property}

	\begin{property}
		Пусть $\xi_1, \ldots, \xi_n$~---независимые случайные величины, $S_n = \sum\limits_{i = 1}^{n}\xi_i \Rightarrow \varphi_{S_n}(t) = \prod\limits_{i = 1}^n\varphi_{\xi_k}(t).$  
		\begin{proof}
			$\varphi_{S_n}(t) = \E e^{it \sum\limits_{k = 1}^{n}\xi_k} = \E\prod\limits_{k = 1}^n e^{it\xi_k} =$ 
			
			/$e^{it\xi_k}$ - борелевская функция от независимой случайной величины. / 
			
			$= \prod\limits_{k = 1}^{n}\E e^{it\xi_k} = \prod\limits_{k = 1}^{n}\varphi_{\xi_k}(t).$
		\end{proof}
	\end{property}
	\begin{property}
		Пусть $\varphi(t)$~--- характеристическая функция, тогда $\varphi(t) = \overline{\varphi(-t)}$
		\begin{proof}
			$\varphi(t) = \E e^{it\xi} = \E\overline{e^{-it\xi}} = \overline{\E e^{-it\xi}} = \overline{\varphi(-t)}.$
		\end{proof}
	\end{property}
	\begin{property}
		Пусть $\varphi(t)$~--- характеристическая функция случайной величины $\xi$, тогда $\varphi(t)$ равномерно непрерывна на $\mathbb{R}.$
		\begin{proof}
			Рассмотрим $|\varphi(t + h) - \varphi(t)| = |\E e^{i(t+h)\xi} - \E e^{it\xi}| = |\E e^{it\xi}(e^{ih\xi} - 1)| \leqslant \E |e^{it\xi}|\cdot |e^{ih\xi} - 1| = \E |e^{ih\xi} - 1|.$
			При $h \to 0$ выполнено $e^{ih\xi} - 1 \overset{\text{п.н.}} \longrightarrow 0 $ по теореме о наследовании сходимости. $\forall h: |e^{ih\xi} - 1| \leqslant |e^{ih\xi}| + 1 = 2, ~ \E 2 < +\infty$. Следовательно, по теореме Лебега о мажорируемой сходимости, $\E |e^{ih\xi} - 1| \to \E 0 = 0$. Значит, $\varphi(t)$ равномерно непрерывна.
		\end{proof}
	\end{property}
		\begin{theorem}[единственности (д-во позже)]
		Пусть $F$ и $G$~--- функции распределения, такие что $\varphi_{F(x)} = \varphi_{G(x)} \Rightarrow \forall x\  F(x) = G(x).$
	\end{theorem}
	
	\begin{property}
		Пусть $\varphi_\xi(t)$~--- характеристическая функция случайной величины $\xi$. $\varphi(t)$ принимает действительные значения $\Leftrightarrow \xi$ имеет симметричное распределение.
		\begin{proof}
			$~(\Leftarrow)$ Пусть распределение $\xi$~---симметрично, тогда $E(\sin(t\xi)) = E(\sin(t(-\xi))) = -E(\sin(t\xi)) = 0.$ Значит $\varphi_\xi(t) = E\cos t\xi + iE\sin t\xi = E\cos t\xi \in \mathbb{R}.$

			$(\Rightarrow)$ Пусть $\varphi_\xi(t) \in \mathbb{R} ~\forall t.$ Тогда по свойствам $2$ и $4$ $\varphi_\xi(t) = \overline{\varphi_\xi(-t)} = \varphi_\xi(-t) = \varphi_{-\xi}(t) ~\Rightarrow $~$\xi$ и $-\xi$ имеют одинаковую характеристическую функцию $~\Rightarrow$~ $\xi \overset{d}{=} -\xi$ по теореме единственности.
		\end{proof}
	\end{property}

	\begin{property}
	\begin{theorem}[о производных х.ф.]
		Пусть $E|\xi|^n < + \infty, ~n \in \mathbb{N}.$ 
		
		Тогда $\forall k \leqslant n~ \exists \varphi_\xi^{(k)}(t),$ причём
		\begin{enumerate}
			\item $\varphi_\xi^{(k)}(t) = \int\limits_\mathbb{R}(ix)^ke^{itx}dF(x)$
			\item $E\xi^k = \frac{\varphi_\xi^{(k)}(0)}{i^k}$
			\item $\varphi_\xi(t) = \sum\limits_{k = 0}^{n} \frac{(it)^k}{k!}E\xi^k + \frac{(it)^n}{n!}\varepsilon_n(t)$, где
		\end{enumerate}
		$|\varepsilon_n(t)| \leqslant 3E|\xi|^n, ~\varepsilon_n(t) \to 0, ~ t \to 0.$
	\end{theorem}
	\begin{proof}
		\begin{enumerate}
			\item Рассмотрим $\frac{\varphi_\xi(t + h) - \varphi_\xi(t)}{h} = \frac{Ee^{i(t + h)\xi} - Ee^{it\xi}}{h} = \frac{Ee^{it\xi}(e^{ih\xi} - 1)}{h}.$ при $h \to 0$ $\frac{e^{ih\xi} - 1}{h} \overset{\text{п.н.}}{\longrightarrow}i\xi,$ кроме того, $\left|\frac{e^{ih\xi} - 1}{h}\right| \leqslant |\xi|$ почти наверное, так как хорда меньше дуги. По теореме о мажорируемой сходимости
			$\lim\limits_{h \to 0}E \frac{e^{ih\xi} - 1}{h}e^{it\xi} = \varphi'_\xi(t) = E(i\xi\cdot e^{it\xi}) = \int\limits_\mathbb{R} ixe^{itx}dF_\xi(x).$ Доказательство формулы для $\varphi^{(k)}$ аналогично.
			\item Из пункта 1, $E\xi^n = \int\limits_\mathbb{R}x^k dF_\xi(x) = \frac{1}{i^k} \int\limits_\mathbb{R}(ix)^ke^{i0x}dF(x) = \frac{\varphi^{(k)}(0)}{i^k}.$
			\item Ряд Тейлора $e^{i\eta} = \sum\limits_{k = 0}^{n - 1} \frac{(i\eta)^k}{k!} + \frac{(i\eta)^n}{n!}(\cos\theta_1y + i\sin\theta_2 y), ~|\theta_1| \leqslant 1, ~ |\theta_2| \leqslant 1,$ тогда
			$\varphi_\xi(t) = Ee^{it\xi} = E\left[\sum\limits_{k = 0}^{n - 1} \frac{(it\xi)^k}{k!} + \frac{(it\xi)^n}{n!}(\cos\theta_1 t\xi + i\sin \theta_2 t\xi)\right] = \sum\limits_{k = 0}^{n}\frac{(it)^k}{k!}E\xi^k + \frac{(it)^n}{n!}\varepsilon_n(t), $ где $\varepsilon_n(t) = E(\xi^n\cdot [\cos\theta_1t\xi + i\sin(\theta_2 t\xi) - 1]) \Rightarrow \varepsilon_n(t) \leqslant 3E|\xi|^n;$

			$|\xi^n[\cos(\theta_1t\xi) + i\sin(\theta_2t\xi) - 1]| \leqslant3|\xi|^n$ и $\xi^n(\cos(\theta_1t\xi) - 1 + \underbrace{\sin(\theta_2 t\xi)}_{\to 0}) \overset{\text{п.н.}}{\longrightarrow} 0$ при $t \to 0 \Rightarrow$ по теореме Лебега о мажорируемой сходимости, $\varepsilon_n(t) \underset{t \to 0}{\longrightarrow} 0.$
		\end{enumerate}
	\end{proof}
	\end{property}
	\begin{property}[б/д]
		Если существует и конечна $\varphi^{(2n)}(0),$ то $E|\xi|^{2n} < +\infty.$
	\end{property}
	\begin{theorem}[о разложении х.ф. в ряд]
		Пусть $\xi$ случайная величина, такая что $E|\xi|^n < +\infty~ \forall n.$ 
		
		Если для некоторого $T > 0 \hookrightarrow \overline{\lim\limits_{n}}\left(E \frac{|\xi|^n}{n!}\right) < \frac{1}{T},$ то $\forall t: |t| < T$ выполнено $\varphi_\xi(t) = \sum\limits_{n = 0}^{+\infty} \frac{(it)^n}{n!}E\xi^n.$
		\begin{proof}
			Пусть $t_0$ такое, что $|t_0| < T, $ тогда $\overline{\lim\limits_{n \to +\infty}}E\left(\frac{|\xi|^n \cdot |t_0|^n}{n!}\right)^{\frac{1}{n}} = \frac{|t_0|}{T} < 1, $ следовательно, по признаку Коши-Адамара сходимости рядов, ряд $\sum\limits_{n = 0}^{+\infty} \frac{E|\xi|^n\cdot|t_0|^n}{n!}$ сходится.
			Рассмотрим $|t| \leqslant |t_0|: \varphi_\xi(t) = \sum\limits_{k = 0}^{n} \frac{(it)^k}{k!}E\xi^k + \underbrace{\frac{(it)^n}{n!}\varepsilon_n(t)}_{R_n(t)}~~~(*).$ 

			\noindent$R_n(t) \leqslant 3\cdot \frac{|t|^n}{n!}\cdot E|\xi|^n \underset{n \to +\infty}{\longrightarrow} 0$ по условию теоремы. Устремляя $n\to + \infty$ в $(*)$, получаем $\varphi_\xi(t) = \sum\limits_{k = 0}^{+\infty} \frac{(it)^k}{k!}E\xi^k.$ В силу произвольности $|t_0| < T,$ разложение верно $\forall t \in (-T, T).$
		\end{proof}
	\end{theorem}

	\begin{example} (Харфункция нормального распределения)
	
		Пусть $\xi \sim N(0;1) \Rightarrow \varphi_\xi(t) = e^{- \frac{t^2}{2}}.$ Мы знаем, что $E\xi^m = \left\{
			\begin{array}{l}
			(m-1)!!, ~m \vdots 2\\
			0, ~m\not \vdots 2
			\end{array}
		\right.$
		$E|\xi|^m = \left\{
			\begin{array}{l}
			(m - 1)!!, ~m \vdots 2\\
			(m - 1)!!\sqrt{\frac{2}{\pi}}, m \not \vdots 2
			\end{array}
		\right. \Rightarrow$ по предыдущей теореме:
		
		$\varphi_\xi(t) = \E e^{it\xi} = \sum\limits_{n = 0}^{\infty} \frac{(it)^{2n}}{(2n)!}(2n - 1)!! = \sum\limits_{n = 0}^{+\infty} \frac{(-t)^{2n}}{(2n)!!} = \sum\limits_{n = 0}^{+\infty} \dfrac{(- t^2)^n}{(2n)!!} = \sum\limits_{n = 0}^{+\infty} \left(- \frac{t^2}{2}\right)^n \frac{1}{n!} = e^{-\frac{t^2}{2}}.$

		\underline{\text{Условие теоремы}}: $\left(\frac{E|\xi|^m}{m!}\right)^{\frac{1}{m}} \leqslant \left(\frac{(m - 1)!!}{m!}\right)^{\frac{1}{m}} = \left(\frac{1}{m!!}\right)^{\frac{1}{m}} \approx \frac{1}{\sqrt{2}}\left(\left(\frac{m}{2e}\right)^{\frac{m}{2}}\right)^{\frac{1}{m}} \sim \frac{C}{\sqrt{m}} \to 0 \Rightarrow T = + \infty.$ (В приближении воспользовались формулой Стирлинга.)
	\end{example}

	\begin{theorem}[формула обращения (б/д)]
		Пусть $\varphi(t)$ характеристическая функция функции распределения $F$. Тогда
		\begin{enumerate}
			\item Для $\forall a < b$ (точки непрерывности) $F$ выполнено
			$F(b) - F(a) = \frac{1}{2\pi} \lim\limits_{c \to +\infty} \int\limits_{-c}^{c} \frac{e^{-itb} - e^{-ita}}{-it}\varphi(t)dt$
			\item Если $\int\limits_\mathbb{R}|\varphi(t)|dt < +\infty,$ то у функции распределения $F(x)$ существует плотность $f(x)$ и $f(x) = \frac{1}{2\pi}\int\limits_\mathbb{R} e^{-tx}\varphi(t)dt.$
		\end{enumerate}
	\end{theorem}
	
	\begin{theorem}[единственности]
		Пусть \(F\) и \(G\)~--- функции распределения, такие что \(\varphi_{F(x)} = \varphi_{G(x)} \Rightarrow \forall x~F(x) = G(x.\)
	\end{theorem}
	\begin{proof}
		Пусть \(a < b \in \mathbb{R}\). Рассмотрим \(f_\varepsilon(x)\)(шапочка). Докажем, что \(\forall \varepsilon > 0 \int\limits_{\mathbb{R}} F_\varepsilon(x) df(x) = \int\limits_\mathbb{R}f_\varepsilon(x)dG(x).\) Рассмотрим отрезок \([-n, n]\) такой, что \([a, b + \varepsilon] \subset [-n, n].\) По теореме Вейерштрасса-Стоуна (приближение любой функции тригонометрическими полиномами), \(f_\varepsilon(x)\) сколь угодно точно приближается тригонометрическими многочленами от \(\frac{\pi x}{n}\), так как \(f_\varepsilon(x)\) непрерывна и периодична на \([-n, n]\) с периодом \(2n\) на \(\mathbb{R}\).

		\noindent \(\Rightarrow \forall n ~ \exists f_\varepsilon^n(x) = \sum\limits_{k \in K} a_k\cdot e^{\frac{ik\pi x}{n}}, ~a_k \in \mathbb{R},\) \(K\)~--- конечное подмножество \(\mathbb{Z},\) такое, что \(\forall x \in [-n, n]: |f_\varepsilon^n(x) - f_\varepsilon(x)| < \frac{1}{n}.\) \(f_\varepsilon^n\)~---периодическая с периодом \(2n\) на \(\mathrm{R}\). Поскольку \(|f_\varepsilon(x)| < 2\) и \(\forall x \in [-n, n]: |f_\varepsilon^n(x) - f_\varepsilon(x)| < \frac{1}{n}\), то \(|f_\varepsilon^n(x)| \leqslant 2 ~ \forall x.\) По условию, \(\forall t~\int\limits_\mathbb{R} e^{itx}dF(x) = \int\limits_\mathbb{R} e^{itx}dG(x) \Rightarrow \int\limits_\mathbb{R} f_\varepsilon^n(x)dF(x) = \int\limits_\mathbb{R} f_\varepsilon^n(x)dG(x).\)
		\begin{gather*}
			\left|\int\limits_\mathbb{R} f_\varepsilon(x) dF(x) - \int\limits_\mathbb{R} f_\varepsilon(x)dG(x)\right| \leqslant \left|\int\limits_\mathbb{R} f_\varepsilon(x) dF(x) - \int\limits_\mathbb{R} f_\varepsilon^n(x) dF(x)\right| +\\
			+  \left|\int\limits_\mathbb{R} f_\varepsilon^n(x) dF(x) - \int\limits_\mathbb{R} f_\varepsilon^n(x) dG(x)\right| + \left|\int\limits_\mathbb{R} f_\varepsilon^n(x) dG(x) - \int\limits_\mathbb{R} f_\varepsilon(x) dG(x)\right| \leqslant\\
			\leqslant \frac{1}{n} \int\limits_{[-n,n]}dF(x) + \frac{1}{n} \int\limits_{[-n,n]}dG(x) +(\underbracket{1 - F(n)}_{\to 0} + \underbracket{F(-n)}_{\to 0} + \underbracket{1 - G(n)}_{\to 0} + \underbracket{G(-n)}_{\to 0}) \leqslant\\
			\leqslant \frac{2}{n} + o(1) \Rightarrow \forall \varepsilon> 0: \int f_\varepsilon(x) dF(x) = \int f_\varepsilon(x)dG(x).
		\end{gather*}
		При \(\varepsilon \to 0 f_\varepsilon(x) \to I_{[a,b]}(x),\) при этом \(|f_\varepsilon(x)| \leqslant 1~ \forall x \in \mathbb{R}.\) По теореме Лебега о мажорировании сходимости(рассматриваем \(f_\varepsilon(x)\) как набор случайных величин на \((\mathbb{R}, \B(\R), P_f) \to (\mathbb{R}, \B(\mathbb{R}))\)).
		\(\int\limits_\R f_\varepsilon(x)dF(x) \to \int\limits_\R I_{[a,b]}dF(x) = F(b) - F(a).\)
		Аналогично, для функции распределения \(G\) \(\int\limits_\R f_\varepsilon(x)dG(x) \underset{\varepsilon \to 0}{\longrightarrow} G(b) - G(a) \Rightarrow \forall a < b ~ F(b) - F(a) = G(b) - G(a).\)  Полагая \(a = (-\infty), \) получаем требуемое.
	\end{proof}

	\begin{theorem}[критерий назависимости]
		Пусть \(\vec{\xi} = (\xi_1, \ldots, \xi_n).\) Тогда \((\xi_1, \ldots, \xi_n)\)~--- назависимые в совокупности \(\Leftrightarrow \varphi_{\vec{\xi}(\vec{t})} = \prod\limits_{i = 1}^n \varphi_{\xi_i}(t_i) ~\forall \vec{t} = (t_1, \ldots, t_n) \in \R^n.\)
	\end{theorem}

	\begin{proof}
		\((\Rightarrow) ~ \varphi_{\vec{\xi}(\vec{t})} = Ee^{i(\vec{t}, \vec{\xi}}) = Eei^{\sum\limits_{k = 1}^{n}t_k\xi_k} \overset{\text{нез-сть}}{=} \prod\limits_{k = 1}^n Ee^{it_k\xi_k} = \prod\limits_{k = 1}^n \varphi_{\xi_x}(t_k).\)

		\((\Leftarrow)\) Пусть \(F_k(x)\)~--- функция распределения случайной величины \(\xi_k.\) Пусть \(G(x_1, \ldots, x_n) = F_1(x)\cdot\ldots\cdot F_n(x)\)~--- это функция распределения. Посчитаем её характеристическую функцию:
		\(\varphi_G(t) = \int\limits_{\R^n}e^{i(\vec{t}, \vec{x})}dG(\vec{x}) = \int\limits_\R e^{i(\vec{t}, \vec{x})}dF_1(x_1)\cdot\ldots\cdot dF_n(x_n) = \)(по теореме Фубини) \(\prod\limits_{k = 1}^n \int\limits_\R e^{it_kx_k} dF_k(x_k) = \prod\limits_{k = 1}^n\varphi_{\xi_k}(t_k) \overset{\text{по усл}}{=}\varphi_{\vec{\xi}}(\vec{t}) \Rightarrow\) характеристическая функция \(G\) и \(\vec{\xi}\) совпадают \(\Rightarrow\) по теореме единственности \(F_\xi = G \Rightarrow F_{\vec{\xi}}(\vec{x}) = \prod\limits_{k = 1}^n F_{\xi_k}(x_k) \Rightarrow \xi_1, \ldots, \xi_n\) независимы в совокупности по критерию независимости в терминах функции распределения.
	\end{proof}

	\subsection{Проверка того, что \(\varphi\)~---характеристическая функция}

	\begin{definition}
		Функция \(\varphi(t)\) является неотрицательно определённой, если \(\forall n ~ \forall t_1, \ldots, t_n \in \R, ~\forall z_1, \ldots, z_n \in \mathbb{C}, ~ \sum\limits_{i, j = 1}^{n}\varphi(t_i - t_j)z_i\overline{z_j} \geqslant 0\).
	\end{definition}

	\begin{theorem}[Бохнера-Хинчина]
		Пусть \(\varphi(t)\) такая, что \(\varphi(0) = 1\) и \(\varphi(t)\) непрерывна в нуле. 
		
		Тогда \(\varphi(t)\)~--- характеристическая функция \(\Leftrightarrow \varphi(t)\) неотрицательно определённая.
	\end{theorem}

	\begin{proof}
		\((\Rightarrow) ~ \varphi(t)\)~--- характеристическая функция, проверим неотрицательность:
		\begin{gather*}%тут не получается написать нормальный символ матожидания
			\forall t_1, \ldots, t_n \in \R~\forall z_1, \ldots, z_n \in \mathbb{C}\\
			\sum\limits_{j, k = 1}^{n}\varphi(t_j - t_k)z_j \overline{z_k} =
			\sum\limits_{j, k = 1}^{n}Ee^{i(t_j - t_k)\xi}z_j\overline{z_k} = E\bigg(\Big(\sum\limits_{j = 1}^ne^{it_j\xi}z_j\Big)\overline{\Big(\sum\limits_{k = 1}^ne^{it_k\xi}z_k\Big)}\bigg) =\\
			= E \sum\limits_{k, k = 1}^{n}e^{it_j\xi}\cdot z_j \cdot \overline{e^{et_k\xi}}\cdot \overline{z_k} = E\left|\sum\limits_{j = 1}^{n} e^{it_j\xi}z_j\right|^2 \geqslant 0
		\end{gather*}
	
		\((\Leftarrow)\) [б/д]
	\end{proof}
	
	\begin{consequence}
		Если \(\varphi(t) = \psi(t)\)~--- характеристическая функция, \(\alpha \in (0, 1),\) то \(\alpha \varphi(t) + (1 - \alpha)\psi(t)\)~--- характеристическая функция.
	\end{consequence}
	
	\begin{proof}
		Все три условия из теоремы Бохнера-Хинчина выполнены.
	\end{proof}
	
	\begin{theorem}[Пойа(б/д)]
		Пусть непрерывная, чётная и выпуклая вниз на \((0; + \infty)\) функция \(\varphi(t)\) такова, что \(\varphi(t) \geqslant 0, ~ \varphi(0) = 1, ~\varphi(t) \underset{t \to +\infty}{\longrightarrow} 0\). 
		
		Тогда \(\varphi(t)\)~---характеристическая функция.
	\end{theorem}

	\begin{example}
		Любая функция вида
		\begin{center}
			\includegraphics[scale=0.4]{img/img1.png}
		\end{center}
		является характеристической.
	\end{example}

	\begin{theorem}[Марцинкевича(б/д)]
		Если характеристическая функция \(\varphi(t)\) имеет вид ~\(\exp(P(t)),\) где \(P(t)\)~--- полином, то степень этого полинома \(\leqslant 2 ~(\deg P(t) \leqslant 2).\)
	\end{theorem}
	\begin{example}
		\(e^{-t^n}\) не является характеристической функцией.
	\end{example}

	\begin{definition}
		Последовательность функций \(F_n(x)\) слабо сходится к \(F(x)\), если \(\forall f(x)\)~---  непрерывна и ограничена, то верно \(\int\limits_\R f(x) dF_n(x) \to \int\limits_\R f(x)dF(x).\) Обозначение \(F_n \overset{w}{\longrightarrow}F.\)
		\((\xi_n \overset{d}{\longrightarrow}\xi \Leftrightarrow F_{\xi_n} \overset{w}{\longrightarrow}F).\)
	\end{definition}
	\begin{theorem}[непрерывности для х.ф.]~~~~~~~~~~~~~~~~~~~~~~~~~~~~~~~~~~~~~~~~~

		\noindent1. ~Пусть \(\{F_n\}_{n \geqslant 1}\)~--- последовательность функций распределения на \(\R\), \({\varphi_n(t)}_{n \in \mathbb{N}}\) --- их х.ф. 
		
		Тогда  \(F_n \overset{w}{\longrightarrow}F \Rightarrow \forall t \in \R~\varphi_n(t) \to \varphi(t)\), где \(\varphi\)~--- характеристическая функция \(F\).
		
		\noindent2.(б/д)~ Пусть \(\forall t \in \R ~ \exists \varphi(t) = \lim\limits_{n \to +\infty} \varphi_n(t), \) причём \(\varphi(t)\) непрерывна в нуле. 
		
		Тогда \(\exists F\)~--- функция распределения такая, что \(F_n \overset{w}{\longrightarrow}F\) и \(\varphi(t)\)~---характеристическая функция \(F\).
	\end{theorem}

	\begin{proof}
		\(F_n \overset{w}{\longrightarrow} F \), значит \(\forall f\)~--- непрерывной ограниченной функции : \(\int\limits_\R f(x)dF_n(x) \to \int\limits_\R f(x)dF(x).\) Но функции \(\sin tx \) и \(\cos tx\) непрерывны и ограничены \(\Rightarrow\) \(\varphi_n(t) = \int\limits_\R (\cos tx + i\sin tx)dF_n(x) \underset{n \to +\infty}{\longrightarrow} \int\limits_\R (\cos tx + i\sin tx)dF(x) = \varphi(t).\)
	\end{proof}

	\subsection{Центральная предельная теорема}
	\begin{theorem}[ЦПТ в форме Леви]
		Пусть \(\{\xi_n\}_{n\geqslant 1}\)~--- независимые одинаково распределённые случайные величины, \(0 < D\xi < +\infty.\)
		Обозначим \(S_n = \sum\limits_{i = 1}^{n}\xi_i.\) 
		
		Тогда 
		\(\frac{S_n - ES_n}{\sqrt{DS_n}} \overset{d}{\underset{n \to +\infty}{\longrightarrow}}N(0, 1).\)
	\end{theorem}

	\begin{proof}
		Обозначим \(E\xi_i = a, D\xi_i = \sigma^2.\)
		Рассмотрим случайные величины \(\eta_i = \frac{\xi_i - a}{\sigma} \Rightarrow 
		E\eta_i = 0; ~ D\eta_i = 1.\)
		Тогда \(T_n = \frac{S_n - ES_n}{\sqrt{DS_n}} = \frac{S_n - na}{\sqrt{n\sigma^2}} = \frac{\eta_1 + \ldots + \eta_n}{\sqrt{n}}.\)
		Рассмотрим характеристическую функцию \(\eta_i\): по свойствам характеристической функции о разложении в ряд \(\varphi(t) \equiv \varphi_{\eta_i}(t) = 1 + it\underbracket{E\eta_j}_{0} + \frac{1}{2}\underbracket{E\eta_j^2}_1\cdot(it)^2 + o(t^2 = 1 - \frac{t^2}{2} + o(t^2), ~ t \to 0.\) Отсюда, \(\varphi_{T_n}(t) = \varphi_{\sum\limits_{j = 1}^{n}\eta_j}(t) \overset{\text{св-ва х.ф. о независимости}}{=} \left(\varphi\left(\frac{t}{\sqrt{n}}\right)\right)^n = \left(1 - \frac{t^2}{2n} + o\left(\frac{t^2}{n}\right)\right)^n \underset{n \to \infty}{=} e^{- \frac{t^2}{2}} \forall t. \)Но \(e^{-\frac{t^2}{2}}\)~--- характеристическая функция \(N(0,1) \Rightarrow\) (по т. непрерывности) \(T_n = \frac{S_n - ES_n}{\sqrt{DS_n}} \overset{d}{\longrightarrow} N(0,1).\)
	\end{proof}
	
	\begin{theorem}[Линдберга][б/д]
		Пусть 
		\begin{enumerate}
		    \item случайные величины \(\{\xi_k\}_{k \geqslant 1}\) независимы, и \(\forall k~ E\xi_k^2 < +\infty\).
		    \item Обозначим \(m_k = E\xi_k; ~\sigma^2_k = D\xi_k; ~S_n = \sum\limits_{i = 1}^{n}\xi_i; ~D_n^2 = \sum\limits_{k = 1}^{n}\sigma^2_k\) и \(F_k(x)\)~--- функция распределения \(\xi_k\).
		    \item Пусть выполнено условие Линдберга:
    		\[
    			\forall \varepsilon > 0 ~ \frac{1}{D^2_n} \sum\limits_{k = 1}^{n} \int\limits_{\{x:|x - m_k| > \varepsilon D_n\}}(x-m_k)^2 dF_k(x) \underset{n \to \infty}{\longrightarrow} 0.
    		\]
		\end{enumerate} 
		
		Тогда \( \dfrac{S_n - ES_n}{\sqrt{DS_n}} \overset{d}{\longrightarrow} N(0,1), n\to \infty\).
	\end{theorem}

	\subsection{Когда выполнены условия Линдберга?}

	\begin{enumerate}
		\item Пусть для некоторого \( \delta > 0\) выполнено условие Ляпунова: 
		\[
			\frac{1}{D_n^{2 + \delta}} \sum\limits_{i = 1}^{n} E|\xi_k - m_k|^{2 + \delta} \underset{n \to \infty}{\longrightarrow} 0
		\]
		тогда выполнено условие Линдберга.
		\begin{proof}
		    Пусть фиксировано \( \varepsilon > 0. \)
			\begin{gather*}
				E|\xi_k - m_k|^{2 + \delta} = \int\limits_\R |x - m_k|^{2 + \delta} dF_k(x) \geqslant \\
				\geqslant \int\limits_{|x - m_k| \geqslant \varepsilon D_n} |x - m_k|^{2 + \delta} dF_k(x) \geqslant \varepsilon^\delta D_n^\delta \int\limits_{|x - m_k| > \varepsilon D_n} |x -m_k|^2 dF_k(x)\\
				\Rightarrow \frac{1}{D^{2 + \delta}} \sum\limits_{k = 1}^{n} E|\xi_k - m_k|^{2 + \delta} \geqslant \frac{\varepsilon^\delta}{D^2_n} \sum\limits_{k = 1}^{n} \int\limits_{\{x:|x - m_k| > \varepsilon D_n\}}|x - m_k|^2 dF_k(x).
			\end{gather*}
		\end{proof}
		\item Из условий теоремы Леви вытекает условие Линдберга.
		\begin{proof}
			Пусть \(\{\xi_k\}_{k \geqslant 1}\)~--- независимые одинаково распределённые случайные величины, \(+\infty > D\xi_1 = \sigma^2 > 0, ~E\xi_1 = a \Rightarrow \)
			\begin{gather*}
				\frac{1}{D_n^2} \sum\limits_{k = 1}^{n} \int\limits_{\{x:|x-a|>\varepsilon D_n\}} |x - a|^2 dF_k(x)=\\
				= \frac{1}{n\sigma^2} \sum\limits_{k = 1}^{n} \int\limits_{\{x:|x-a| > \varepsilon D_n\}} |x-a|^2dF_1(x) = \\
				\frac{1}{\sigma^2} \int\limits_{|x-a| > \varepsilon \sqrt{n}\sigma} |x-a|^2 dF_1(x) \to 0, \text{ т.к. }\{x:|x-a| > \varepsilon\sqrt{n}\sigma\} \to \varnothing;\\
				\int\limits_\R |x-a|^2 dF_1(x) < +\infty.
			\end{gather*}
		\end{proof}
		\item Пусть \(\{\xi_k\}_{k \geqslant 1}\)~--- независимые случайные величины, \(|\xi_k| \leqslant K; ~D_n \to +\infty\). Тогда 
		\begin{gather*}
			\int\limits_{|x-m_k| > \varepsilon D_n}(x - m_k)^2 dF_k(x) = \\
			= E((\xi_k - m_k)^2\cdot T(|\xi_k - m_k| > \varepsilon D_n)) \leqslant (2k)^2 EI(|\xi_k - m_k| > \varepsilon D_n) =\\
			=  (2k)^2P(|\xi_k - m_k| > \varepsilon D_n),
		\end{gather*}
		то по неравенству Чебышева это не превосходит \[(2k)^2 \frac{\sigma_k^2}{\varepsilon^2 D_n^2}.\] Рассмотрим сумму \[\frac{1}{D_n^2} \sum\limits_{k = 1}^{n} \int\limits_{x: |x- m_k| > \varepsilon D_n} |x-m_k|^2 dF_k(x) \leqslant \frac{(2k)^2}{D_n^2} \sum\limits_{k = 1}^{n} \frac{\sigma^2_k}{\varepsilon^2 D_n^2} = \frac{(2k)^2}{\varepsilon^2D_n^2} \overset{n \to \infty}{\to 0} \text{ т.к. } D_n \to \infty.\]
	\end{enumerate}

	\begin{note}[Теорема Феллера.]
		Условие Линдберга является необходимым и достаточным условием для справедливости ЦПТ. При выполнении условия бесконечной малости слагаемых:
		\[
			\max\limits_{a < x \leqslant n}P\left(\frac{|\xi_k - m_k|}{D_n} \geqslant \varepsilon\right) \to 0 \text{ при } n \to \infty.
		\]
	\end{note}
	\begin{theorem}[Берри-Эссена(б/д)]
		Пусть \(\{\xi_k\}_{k \geqslant 1}\)~--- независимые одинаково распределённые случайные величины, \(E|\xi_i|^3 < +\infty, ~ E\xi_i = a, ~ D\xi_i = \sigma^2, ~S_n = \sum\limits_{i = 1}^{n}\xi_i; ~ T_n = \dfrac{S_n - ES_n}{\sqrt{DS_n}}.\) Тогда 
		\[
			\sup\limits_{x\in \R}|F_{T_n}(x) - \Phi(x)| \leqslant C\cdot \frac{E|\xi_Ta|^3}{\sigma^3\sqrt{n}}, \text{ где } \frac{1}{\sqrt{2\pi}} < C < 0,48,
		\]
		где \(\Phi(x) = \dfrac{1}{\sqrt{2\pi}}\int\limits_{-\infty}^{x}e^{- \frac{x^2}{2}}dx.\)
	\end{theorem}