\section{Глава 5.}
 \subsection{Виды сходимости случайных величин}
 \begin{definition}
    $\xi_n$ и $\xi$~--- случайные величины. 
 	\begin{enumerate}
 		\item $\xi_n \xrightarrow{\P} \xi$, если 
 			$\forall \varepsilon > 0 \hookrightarrow \P\big(\omega:|\xi_n(\omega) - \xi(\omega)| 
 			\geqslant \varepsilon \big) \limn 0$,
 		\item $\xi_n \xrightarrow{\text{п.н.}} \xi$, если 
 			$\P(\omega: \xi_n \rightarrow \xi) = 1$,
 		\item $\xi_n \xrightarrow{L_p} \xi$, если $\E |\xi_n |^p < + \infty$, $\E |\xi|^p < +\infty$ и $\E | \xi_n - \xi |^p \limn 0$ ($p > 0$),
 		\item $\xi_n \xrightarrow{d} \xi$, если для любой непрерывной ограниченной функции $f(x)$ выполнено $\E f (\xi_n) \limn \E f(\xi)$.
 	\end{enumerate}
 \end{definition}
 
 \begin{theorem}[Александрова][б/д]
 	$\xi_n \xrightarrow{d} \xi$ тогда только тогда, когда 
 	$F_{\xi_n}(x) \xrightarrow{\text{в основном}} F_\xi (x)$, то есть 
 	$F_{\xi_n}(x) \rightarrow F_\xi(x)$ во всех точках непрерывности 
 	функции распределения $F_\xi(x)$. 
 \end{theorem}
 \begin{lemma}[критерий сходимости почти наверное]
 	$\xi_n \xrightarrow{\text{п.н.}} \xi$ тогда и только тогда, когда $\forall \varepsilon > 0 \hookrightarrow \P \big(\omega: \sup\limits_{k \geqslant n} | \xi_k(\omega) - \xi(\omega) | \geqslant \varepsilon \big) \limn 0$.
 	\begin{proof}
 		Пусть $A_k^\varepsilon = 
 		\big\{ \omega: | \xi_k - \xi | \geqslant \varepsilon \big\}$, $A^\varepsilon = 
 		\bigcap\limits_{n=1}^{+\infty} \bigcup\limits_{k \geqslant n} A_k^\varepsilon = 
 		\big\{\omega: \forall n~ \exists k \geqslant n: |\xi_k - \xi| \geqslant \varepsilon \big\}$. 
 		Тогда $\big\{ \omega: \xi_n (\omega) \not\rightarrow \xi(\omega) \big\} = 
 		\bigcup\limits_{m=1}^{+\infty} A^{\frac{1}{m}} = 
 		\big\{\omega: \exists m~ \forall n~ \exists k \geqslant n: | \xi_k(\omega) - \xi(\omega) | > \frac{1}{m} \big\}$. Следовательно, по непрерывности вермеры: 
 		\begin{multline*}
 			\P \big( \omega: \xi_n(\omega) \not\rightarrow \xi(\omega) \big) = 0 \Leftrightarrow \P \left( \bigcup\limits_{m = 1}^{+\infty} A^{\frac{1}{m}} \right) = 0 \Leftrightarrow \\ \Leftrightarrow \forall m \in \N : \P \left( A^{\frac{1}{m}} \right) = 0  \Leftrightarrow \forall \varepsilon > 0 : \P  \left( A^\varepsilon \right) = 0,
 		\end{multline*}
 		так как всегда существует $m$, что $\frac{1}{m} \geqslant \varepsilon \geqslant \frac{1}{m+1}$, то есть $A^{\frac{1}{m+1}} \supseteq A^\varepsilon \supseteq A^{\frac{1}{m}}$. Но $\bigcup\limits_{k \geqslant n} A^\varepsilon_k \downarrow A^\varepsilon$, следовательно, 
 		\begin{multline*}
 			0 = \P \left( A^\varepsilon \right) = 
 			\lim\limits_{n \rightarrow + \infty} \P \left( \bigcup_{k \geqslant n} A^\varepsilon_k \right) \Leftrightarrow
 			\forall \varepsilon > 0 : \P \left( \bigcup\limits_{k \geqslant n} A^\varepsilon_k \right) \limn 0 \Leftrightarrow \\ 
 			\Leftrightarrow  \forall \varepsilon > 0:  \P \big(\omega: \sup\limits_{k \geqslant n} | \xi_k(\omega) - \xi(\omega) | \geqslant \varepsilon \big) \limn 0.
 		\end{multline*}
 	\end{proof}
 \end{lemma}
 
 \begin{theorem}[взаимоотношения различных видов сходимости]
 	~\\ \hspace*{5cm}
 		\begin{tikzpicture}
  		\path ( 0, 1) node (everywhere) {$\text{п.н.}$}
        		( 0, 0) node (lp) {$L_p$}
        		( 1, 0.5) node (probability) {$\P$}
        		( 2, 0.5) node (distribution) {$d$};
  		\draw [->] (everywhere) to (probability);
		\draw [->] (lp) to (probability);
		\draw [->] (probability) to (distribution);
		\end{tikzpicture}
	\begin{proof}
		($\text{п.н.} \Rightarrow \P$)\qquad $\xi_n \xrightarrow{\text{п.н.}} \xi  \Leftrightarrow \forall \varepsilon > 0 : \P \big(\omega: \sup\limits_{k \geqslant n} | \xi_k(\omega) - \xi(\omega) | > \varepsilon \big) \rightarrow 0$,~но 
		$$\big\{\omega: |\xi_n(\omega) - \xi(\omega)| > \varepsilon \big\} \subset \big\{ \omega: \sup\limits_{k \geqslant n} |\xi_n(\omega) - \xi(\omega)| \geqslant \varepsilon \big\},$$
		 следовательно, $\P \big( | \xi_n - \xi | \geqslant \varepsilon \big) \leqslant \P \big(\sup\limits_{k \geqslant n}|\xi_k - \xi| \geqslant \varepsilon \big) \rightarrow 0$.\\
		
		($L_p \Rightarrow \P$)\qquad $\P \big( | \xi_n -\xi | \geqslant \varepsilon \big) = \P\big(\omega: | \xi_n(\omega) - \xi(\omega) |^p \geqslant \varepsilon^p \big)$, а по неравенству Маркова это меньше или равно $\dfrac{\E |\xi_n(\omega) - \xi (\omega)|^p}{\varepsilon^p} \limn 0$.\\
		
		($\P \Rightarrow d$)\qquad Пусть $f(x)$~--- ограниченная непрерывная функция, тогда $\exists C \in \R~\forall x \in \R \hookrightarrow |f(x)| \leqslant C$. Зафиксируем $\varepsilon > 0$, возьмем $N \in \R : \P \big( |\xi | > N \big) \leqslant \dfrac{\varepsilon}{4C}$. На отрезке $[-\N, \N]$ функция $f(x)$ равномерно непрерывна, следовательно, 
		$$\exists \delta > 0~~\forall x, y \in [-\N, \N] \hookrightarrow \left( |x-y| < \delta \Rightarrow |f(x) - f(y) | < \dfrac{\varepsilon}{2} \right).$$
		Рассмотрим разбиение $\Omega$: 
		\begin{align*}
			A_1 &= \big\{\omega: |\xi(\omega)| < N,~ |\xi_n(\omega) - \xi(\omega)| \leqslant \delta \big\},\\
			A_2 &= \big\{\omega: |\xi(\omega)| > N,~ |\xi_n(\omega) - \xi(\omega)| \leqslant \delta \big\},\\
			A_3 &= \big\{\omega:  |\xi_n(\omega) - \xi(\omega)| > \delta \big\}.
		\end{align*}
		Оценим 
		$$\big| \E f(\xi_n) - \E f(\xi) \big| \leqslant \E \big| f(\xi_n) - f(\xi) \big| = \E \big[ | f(\xi) - f(\xi_n)| \cdot ( I_{A_1} + I_{A_2} + I_{A_3} ) \big] \boxed{\leqslant}.$$ 
		Пусть $\omega \in A_1$, тогда, так как $|\xi_n - \xi| \leqslant \delta$, то $|x - y| \leqslant \delta$, а значит $\big| f(\xi_n) - f(\xi) \big| \leqslant \dfrac{\varepsilon}{2}$, следовательно, $\E \big[ |f(\xi_n) - f(\xi) | \cdot I_{A_1} \big] \leqslant \dfrac{\varepsilon}{2} \cdot \E I_{A_1} = \dfrac{\varepsilon}{2} \cdot \P(A_1) \leqslant \dfrac{\varepsilon}{2}$. Если же $\omega \in A_2, A_3$, то   $|f(\xi_n) - f(\xi)| \leqslant 2C$ так как $f$ ограничена.
		
		Значит, $\boxed{\leqslant} \dfrac{\varepsilon}{2} + 2C \cdot \P(A_2) + 2C \cdot \P(A_3) \leqslant \dfrac{\varepsilon}{2} + 2C \cdot \P \big( |\xi| > N \big) + \cancelto{0,~ \text{т.\,к.}~\xi_n \xrightarrow{\P} \xi}{2C \cdot \P \big( |\xi_n - \xi| > \delta \big)} \leqslant~C_1 \varepsilon$, где $\P \big( |\xi| > N \big) \leqslant \dfrac{\varepsilon}{4C}$. Следовательно, в силу произвольности $\varepsilon$, $\E f(\xi_n) \rightarrow \E f(\xi)$, то есть $\xi_n \xrightarrow{d} \xi$. 
	\end{proof}
 \end{theorem}
 
 \subsection{Контрпримеры}
\begin{example}[п.н. $\not\Rightarrow L_p$, а значит, $\P \not\Rightarrow \L_p$ и $d \not\Rightarrow L_p$]
	Рассмотрим $\Omega = [0, 1]$, $\F = \B\big([0,1] \big)$, $\P$ --- мера Лебега на $[0, 1]$. Пусть $\xi_n = e^n \cdot I_{\left[0, \frac{1}{n} \right]}$, тогда $\forall \omega \in (0, 1) \ \exists n : \omega > \dfrac{1}{n} \Rightarrow \forall k \geqslant n \ \xi_k(\omega) = 0$ (значит имеется сходимость п.н.), следовательно $\xi_n \xrightarrow{\text{п.н.}} \xi = 0$, но $\E | \xi_n - \xi |^p = e^{np} \cdot \dfrac{1}{n} \rightarrow + \infty$, значит сходимости в $L_p$ нет.
\end{example}

\begin{example}[$L_p \not\Rightarrow~\text{п.н.}$, $\P \not\Rightarrow~\text{п.н.}$, $d \not\Rightarrow~\text{п.н.}$]
	Рассмотрим $\Omega = [0, 1]$, $\F = \B\big([0,1] \big)$, $\P$ --- мера Лебега на $[0, 1]$. 
	Возьмем $\xi_{2^n + i} = I \left( \omega \in \left[ \dfrac{i}{2^n}, \dfrac{i + 1}{2^n} \right) \right), ~~i = 0,\ldots, 2^n - 1;~~n \in \Z_+$. 
	Тогда $\xi_k \xrightarrow{L_p} 0$ при $k \rightarrow + \infty$, так как $\E | \xi_k  - 0 |^p = \dfrac{1}{2^n} \cdot 1^p \rightarrow 0$, где $n = \left[ \log_2 k \right]$. 
	Но $\forall \omega$ из $[0,1]$ $\exists$ бесконечно много $\xi_i$ таких, что $\xi_i(\omega) = 1$ и $\xi_i(\omega) = 0$, следовательно, $\forall \omega \hookrightarrow \xi_i(\omega) \nrightarrow 0$, ровно как и к 1, в смысле почти наверное.
\end{example}

\begin{example}[$ d \not\Rightarrow \P$] 
	Пусть $\Omega = \{ \omega_1, \omega_2 \}$, $\P(\omega_i) = \dfrac{1}{2}$, $\forall n \in \Z_+ \hookrightarrow \xi_n(\omega_1) = 0, \xi_n(\omega_2) = 1$. Тогда $\xi_n \sim \text{Bern} \left( \frac{1}{2} \right)$. $\xi(\omega_1) = 1, \xi(\omega_2) = 0$, значит, $\xi \sim \text{Bern} \left( \frac{1}{2} \right)$, следовательно, по теореме Александрова $\xi_n \xrightarrow{d} \xi$, но $\P \big( | \xi_n - \xi| > 0.5 \big) = 1$, значит, $\xi_n \not\xrightarrow{\P} \xi$. 
\end{example}

\begin{definition}
    Последовательность случайных величин $\{ \xi_n \}$ фундаментальна почти наверное, если $\P  \big( \omega : | \xi_n(\omega) - \xi_m(\omega) | \rightarrow 0 \big) = 1$ при $n, m \rightarrow + \infty$.
\end{definition}

\begin{lemma}[критерий фундаментальности почти наверное][б/д]
	Последовательность случайных величин $\{ \xi_n \}$ фундаментальна почти наверное тогда и только тогда, когда $\forall \varepsilon > 0 \hookrightarrow \P\big(\omega: \sup\limits_{k \geqslant n} | \xi_k(\omega) - \xi_n(\omega) | \geqslant \varepsilon \big) \limn 0$.
\end{lemma}

\begin{theorem}[критерий Коши сходимость почти наверное]
	Последовательно случайных величин $\{ \xi_n \}$ сходится почти наверное тогда и только тогда, когда $\{ \xi_n \}$ фундаментальна почти наверное.
	\begin{proof}
		($\Rightarrow$)\quad Пусть $\xi_n \xrightarrow{\text{п.н.}} \xi$, тогда, если $\omega \in \big\{ \omega: \xi_n(\omega) \rightarrow \xi(\omega) \big\}$, то по критерию Коши для числовых последовательностей $\omega \in \big\{ \omega: \{\xi_n\}~\text{--- фундаментальная} \big\}$, следовательно, $\P \big( \omega: \{ \xi_n(\omega)\}~\text{--- фундаментальная} \big) \geqslant \P \big( \omega: \xi_n(\omega) \rightarrow \xi(\omega) \big) = 1$.\\
		
		($\Leftarrow$) \quad Обозначим $A = \{ \omega: \{\xi_n\}~\text{--- фундаментальная} \big\}$. Построим такую случайную величину $\xi$, что $\xi_n \xrightarrow{\text{п.н.}} \xi$. По критерию Коши для любого $\omega \in A$ у последовательности $\big\{ \xi_n(\omega) \big\}$ существует предел $\xi(\omega)$. Положим по определению $\xi(\omega) = \lim\limits_{n \rightarrow + \infty} \xi_n(\omega) \cdot I_A(\omega)$. Тогда $\xi_n \cdot I_A \rightarrow \xi$ во всех точках, то есть $\xi$~--- случайная величина, как предел случайных величин, и $\P \big( \omega: \xi_n (\omega) \rightarrow \xi(\omega) \big) = \P(A)=1$.
	\end{proof}
\end{theorem}

\begin{definition}
	Пусть $ \{A_n \}_{n \in \N}$~--- последовательность событий, тогда событием $\{ A_n~\text{бесконечно часто (б.ч.)} \}$ называется событие $\{ \omega: \forall n \exists k \geqslant n: \omega \in A_k \}$, то есть все такие $\omega$, что $\omega$ принадлежит бесконечному числу элементов из $\{ A_n \}_{n \in \N}$. $\{ A_n~\text{б.ч.} \} = \bigcap\limits_{n=1}^{\infty} \bigcup\limits_{k \geqslant n}^{\infty} A_k$.
\end{definition}
\begin{lemma}[Бореля-Кантелли]

	\begin{enumerate}
		\item {Если $\sum\limits_{k = 1}^{\infty} \P(A_k) < +\infty$, то $\P ( A_n~\text{б.ч.}) = 0$.}
		\item {Если $\sum\limits_{k = 1}^{\infty} \P(A_k) = +\infty$ и $\{A_k\}$ независимы в совокупности, то $\P ( A_n~\text{б.ч.}) = 1$.}
	\end{enumerate}
	\begin{proof}
	    \begin{enumerate}
	        \item $\P(A_n~\text{б.ч.}) = \P\left(\bigcap\limits_{n=1}^{\infty} \bigcup\limits_{k \geqslant n}^{\infty} A_k \right) \boxed{=}$. Известно, что $\bigcup\limits_{k \geqslant n} A_n \downarrow \{ A_n~\text{б.ч.}\}$, следовательно, по непрерывности вероятностной меры имеем $\boxed{=} \lim\limits_{n \rightarrow \infty} \P\left( \bigcup\limits_{k \geqslant n} A_k \right) \leqslant \lim\limits_{n \rightarrow \infty} \sum\limits_{k \geqslant n} \P(A_k) = 0$ т.к. ряд сходится.
		
		    \item Заметим, что $\P ( A_n ~\text{б.ч.}) = \P \big( \bigcap\limits_{n=1}^\infty \bigcup\limits_{k \geqslant n} A_k \big) = / \text{по непрерывности вермеры} /= \lim\limits_{ n \rightarrow \infty} \P \left( \bigcup\limits_{k \geqslant n} A_k \right) = 	/ \text{по законам да Моргана} / = \lim\limits_{ n \rightarrow \infty} \left( 1 - \P \left( \bigcap\limits_{k \geqslant n} \overline{A_k} \right) \right)$, (надо доказать, что $\P$ в скобках стремится к нулю). Покажем это:
		    \begin{multline*}
		    	\P \left( \bigcap\limits_{k \geqslant n} \overline{A} \right) =  / \text{непрерывность вермеры}/ = \lim\limits_{N \rightarrow \infty} \P \left( \bigcap\limits_{k = n}^{N} \overline{A_k} \right) = \lim\limits_{N \rightarrow \infty} \prod\limits_{k = n}^{N} \P \left( \overline{A_k} \right) = \\
		        = \lim\limits_{N \rightarrow \infty} \prod\limits_{k = n}^{N} \left[ 1 - \P (A_k) \right] \leqslant / 1- x \leqslant e^{-x} / \geqslant \lim\limits_{N \rightarrow \infty} \prod\limits_{k = n}^{N} \exp\left(-\P \left( \overline{A_k} \right)\right) = \\
		        = \lim\limits_{N \rightarrow \infty} \exp \left( - \sum\limits_{k =n}^{N} \P \left( A_k \right) \right)= \exp \left( - \sum\limits_{k = n}^{\infty}  \P \left( A_k \right) \right) = 0.
		    \end{multline*} 
		   
        В последнем равенстве сумма равна бесконечности, так как это сумма хвоста расходящегося ряда.
        
		Значит, продолжая равенство выше, получаем что $\lim\limits_{n \rightarrow \infty} (1 - 0 ) = 1$.
	    \end{enumerate}
	\end{proof}
\end{lemma}

\begin{definition}
    	Последовательность случайных величин $\{ \xi_n \}$ фундаментальна по вероятности, если 
		$$\forall \varepsilon > 0 \hookrightarrow \P \big(\omega: | \xi_k - \xi_n | > \varepsilon \big) \xrightarrow[n, k \rightarrow \infty]{} 0.$$ 
\end{definition}

\begin{theorem}[Рисса]
	Если последовательность случайных величин $\{ \xi_n \}$ фундаментальна (или сходится) по вероятности, то из нее можно выделить подпоследовательность $\{ \xi_{n_k} \}$ фундаментальную (сходящуюся) почти наверное.
	\begin{proof}
		Т.к. фундаментальность п.н. $\Leftrightarrow$ сходимость п.н., то докажем, что можно выделить подпоследовательность $\{ \xi_{n_k} \}$, сходящуюся почти наверное. Пусть $n_1=1$. По индукции определим $n_k$, как наименьшее $n > n_{k-1}$ такое, что $\forall s \geqslant n, t \geqslant n \hookrightarrow \P \big( | \xi_t - \xi_s | > 2^{-k} \big) < 2^{-k}$. Тогда $\sum\limits_{k = 1}^{\infty} \P \big( | \xi_{n_{k+1}} - \xi_{n_k} | > 2^{-k} \big) < \sum\limits_{k = 1}^{\infty} 2^{-k} < +\infty$, следовательно, по лемме Бореля-Кантелли $\P \big( | \xi_{n_{k+1}} - \xi_{n_k}| > 2^{-k}~\text{б.ч.} \big) = 0$, значит, почти наверное $\sum\limits_{k = 1}^{+\infty} | \xi_{n_{k+1}} - \xi_{n_k} | < + \infty$. Пусть $\mathcal{N} = \left\{ \omega: \sum\limits_{n = 1}^{+\infty} \big| \xi_{n_{k+1}}(\omega) - \xi_{n_k}(\omega) \big| = +\infty \right\}$, тогда $\P(\mathcal{N}) = 0$. Положим $\xi(\omega) =  \left( \xi_{n_1} (\omega) + \sum\limits_{k = 1}^{+\infty} \big(\xi_{n_{k+1}} (\omega) - \xi_{n_k}(\omega) \big) \right) \cdot I(\omega \in \mathcal{N})$, где ряд в скобках сходится на $\omega \in \Omega / \mathcal{N}$.
		Получаем, $\sum\limits_{j = 1}^{k} ( \xi_{n_j + 1} - \xi_{n_j} ) + \xi_{n_1} = \xi_{n_{k+1}} \xrightarrow{\text{п.н.}} \xi$.
		
		Пусть теперь $\xi_n \xrightarrow{\P} \xi$, тогда 
		$$\P \big( | \xi_m - \xi_n| \geqslant \varepsilon \big) \leqslant \P \left( |\xi_n - \xi | \geqslant \dfrac{\varepsilon}{2} \right) + \P \left( |\xi_m - \xi | \geqslant \dfrac{\varepsilon}{2} \right) \xrightarrow[n, m \rightarrow \infty ]{} 0.$$ 
		Следовательно, из сходимости по вероятности следует фундаментальность по вероятности, а дальше все тоже самое (из фундаментальности следует, что можно выделить сходящуюся подпоследовательность.
	\end{proof}
\end{theorem}

\begin{theorem}[критерий Коши сходимости по вероятности]
	$\xi_n \xrightarrow{\P} \xi$ тогда и только тогда, когда $\{\xi_n\}$ фундаментальна по вероятности.
	\begin{proof}
		($\Rightarrow$) \quad Следует из теоремы Рисса.
		
		($\Leftarrow$) \quad Если $\{ \xi_n \}$ фундаментально по вероятности, то по теореме Рисса существует подпоследовательность $\{\xi_{n_k} \}$ такая, что $\xi_{n_k} \xrightarrow{\text{п.н.}} \xi$, то есть из связи между разными видами сходимости: $\xi_{n_k} \xrightarrow{\P} \xi$. Тогда $\P \big( | \xi_n - \xi | \geqslant \varepsilon \big) \leqslant \cancelto{0\text{, т.к. фунд.}}{\P \left( |\xi_n - \xi_{n_k} | \geqslant \dfrac{\varepsilon}{2} \right)} + \cancelto{0\text{, т.к. сход.}}{\P \left( |\xi_{n_k} - \xi | \geqslant \dfrac{\varepsilon}{2} \right)} \limn 0$.
	\end{proof}
\end{theorem}

\begin{theorem}[Неравенство Колмогорова]
	Пусть $\xi_1,  \ldots, \xi_n$~--- независимые случайные величины такие, что
	\begin{enumerate}
	    \item $\E \xi_i = 0$, $\E \xi_i^2 < +\infty$.
	    \item Обозначим $S_n = \sum\limits_{i = 1}^n \xi_i$.
	\end{enumerate}  
	
	Тогда $\forall \varepsilon > 0 \hookrightarrow \P \left( \max\limits_{1 \leqslant k \leqslant n} |S_k| \geqslant \varepsilon \right) \leqslant \dfrac{\E S_n^2}{\varepsilon^2}$.
	\begin{proof}
		Обозначим $A = \{ \max\limits_{1 \leqslant k \leqslant n} |S_k| \geqslant \varepsilon \}$. Разобьем $A$ на несколько непересекающихся событий, то есть $A_k = \big\{ |S_k| \geqslant \varepsilon\big\}$ и $\forall i \leqslant k - 1 \hookrightarrow |S_i| \leqslant \varepsilon$, следовательно, $A = \bigsqcup\limits_{k=1}^{n} A_k$. Тогда
		\begin{multline*}
			\E(S_n^2 \cdot I_{A_k}) = \E\big( (S_k + \underbracket[0.5pt]{\xi_{k+1} + \ldots + \xi_n}_{\overline{S_k}} )^2 \cdot I_{A_k} \big) =  \\= \E( S_k^2 \cdot I_{A_k} ) + \E\left( \overline{S_k}^2 \cdot I_{A_k} \right) + 2 \E\left( S_k \overline{S_k} \cdot I_{A_k} \right) = (*).
		\end{multline*}
		
		Докажем, что $\E \big(S_k\overline{S_k}I_{A_k} \big) = 0$, $I_{A_k}$ зависит от $(S_1, \dots, S_k)$ и не зависит от $(\xi_{k+1}, \dots, \xi_n)$.
		Следовательно, $S_k \cdot I_{A_k} \indep \overline{S_k}$, так как $\{\xi_1, \ldots, \xi_k \} \indep \{ \xi_{k+1}, \ldots,\xi_n \}$, а, значит, $\E( S_k \cdot I_{A_k} \cdot \overline{S_k} ) = \E(S_k \cdot I_{A_k}) \cdot \cancelto{0}{\E \overline{S_k}} = 0$. Отсюда
		\begin{equation*}
			(*) = \E( S_k^2 \cdot I_{A_k} ) + \E \left( \overline{S_k}^2 \cdot I_{A_k}  \right) \geqslant \E(S_k^2 \cdot I_{A_k} ) \geqslant \varepsilon^2 \cdot \E I_{A_k} = \varepsilon^2 \cdot \P(A_k).
		\end{equation*}
		В итоге, 
		\begin{equation*}
			\E S_n^2 \geqslant \E( S_n^2 \cdot I_A ) = \sum\limits_{k = 1}^n \E( S_n^2 \cdot I_{A_k} ) \geqslant \sum\limits_{k = 1}^n \P(A_k) \cdot \varepsilon^2 = \P(A) \cdot \varepsilon^2.
		\end{equation*}
	\end{proof}
\end{theorem}

\begin{theorem}[Колмогорова-Хинчина о сходимости ряда]
	Пусть $\{\xi_n \}_{n \geqslant 1}$~--- последовательность независимых случайных величин такая, что $\E \xi_n = 0 $ и $\E \xi_n^2 < +\infty$. 
	
	Тогда, если $\sum\limits_{n = 1}^{\infty} \E \xi_n^2 < +\infty$, то $ \sum\limits_{n = 1}^{\infty} \xi_n$ сходится почти наверное.
	\begin{proof}
		Обозначим $S_n = \sum\limits_{k=1}^n \xi_k$. По критерию Коши $\left\{ \sum\limits_{n = 1}^{\infty}\xi_n~\text{сходится п.н.} \right\}$ равносильно тому, что $\{ S_n~\text{фундаментально п.н.}\}$, а это в свою по критерию фундаментальности равносильно тому, что 
		$$\forall \varepsilon > 0 \hookrightarrow \P \left( \sup\limits_{k \geqslant n} | S_k - S_n | \geqslant \varepsilon \right) \limn 0.$$ Докажем это.
		 Очевидно, что
		 $$\P \left( \sup\limits_{k \geqslant n} | S_k - S_n | \geqslant \varepsilon \right) = \P \left( \bigcup\limits_{k \geqslant n} \big\{| S_k - S_n | \geqslant \varepsilon \big\} \right) =$$
		 а из непрерывности вероятностной меры следует, что 
		 $$= \lim\limits_{N \rightarrow +\infty} \P \left( \bigcup\limits_{k = n}^{N} \big\{| S_k - S_n | \geqslant \varepsilon \big\} \right) = \lim\limits_{N \rightarrow +\infty} \P \left( \max\limits_{n \leqslant k \leqslant N} | S_k - S_n | \geqslant \varepsilon \right) \leqslant$$
		 А по неравенству Колмогорова:
		 $$\leqslant \lim\limits_{N \rightarrow + \infty} \dfrac{\E ( S_N - S_n)^2}{\varepsilon^2} = \lim\limits_{N \rightarrow + \infty} \dfrac{1}{\varepsilon^2} \E \sum\limits_{k = n + 1}^N \xi_k^2 = $$.
		 Так как $\xi_k$ независимы, то
		 $$= \lim\limits_{N \rightarrow + \infty} \dfrac{1}{\varepsilon^2} \sum\limits_{k = n + 1}^N \E \xi_k^2 = \dfrac{1}{\varepsilon^2} \sum\limits_{k > n} \E \xi_k^2 \limn 0.$$
	\end{proof}
\end{theorem}

\begin{lemma}[Тёплица]
	Пусть $x_n \rightarrow x$~--- числовая последовательность, числа $\{a_n\}_{n \geqslant 1}$ таковы, что $ \forall n \hookrightarrow a_n \geqslant 0$ и $ b_n = \sum\limits_{k = 1}^n a_k \uparrow +\infty$. 
	
	Тогда $\dfrac{1}{b_n} \sum\limits_{i = 1}^n a_i x_i \limn x$.
	\begin{proof}
		Пусть $\varepsilon > 0$. Выберем $n_0$ так, что $\forall n > n_0 \hookrightarrow |x_n - x| \leqslant \dfrac{\varepsilon}{2}$. Выберем $n_1 > n_0$ такое, что $\dfrac{1}{b_{n_1}} \sum\limits_{k=1}^{n_0} a_k | x_k - x| \leqslant \frac{\varepsilon}{2}$, тогда
		\begin{multline*}
			\forall n > n_1 \hookrightarrow \left| \dfrac{1}{b_n} \sum\limits_{k = 1}^n a_k x_k - x \right| = \left| \dfrac{1}{b_n} \sum\limits_{k = 1}^n a_k x_k - \dfrac{1}{b_n} \sum\limits_{k = 1}^n a_k x \right| \leqslant \dfrac{1}{b_n} \sum\limits_{k = 1}^n a_k |x_k - x| = \\= \dfrac{1}{b_n} \sum\limits_{k = 1}^{n_0} a_k |x_k - x| + \dfrac{1}{b_n} \sum\limits_{k = n_0 + 1}^n a_k |x_k - x| \leqslant \dfrac{\varepsilon}{2} + \dfrac{\varepsilon}{2} \cdot \dfrac{1}{b_n} \sum\limits_{k = n_0 + 1}^n a_k \leqslant \varepsilon.
		\end{multline*}
	\end{proof}
\end{lemma}

\begin{lemma}[Кронекера]
	Пусть ряд $\sum\limits_{n = 1}^{\infty} x_n$ сходится, $\{a_n\}_{n \geqslant 1}$ такова, что $a_n \geqslant 0$, $b_n = \sum\limits_{k=1}^n a_k \uparrow + \infty$. 
	
	Тогда $\dfrac{1}{b_n} \sum\limits_{k = 1}^n b_k x_k \limn 0$.
	\begin{proof}
		Пусть $S_n = \sum\limits_{k = 1}^n x_k$, тогда $S_n \limn S = \sum\limits_{k = 1}^\infty x_k$. Воспользуемся методом суммирования Абеля:
		\begin{multline*}
			\sum\limits_{j = 1}^n b_j x_j = \sum\limits_{j = 1}^n b_j (S_j - S_{j - 1}) = b_n S_n - \sum\limits_{j = 1}^n S_{j-1}(b_j - b_{j-1}) = b_n S_n - \sum\limits_{j = 1}^n S_{j-1} a_j.
		\end{multline*}
		Следовательно,
		$$\dfrac{1}{b_n} \sum\limits_{k = 1}^n b_k x_k = S_n - \cancelto{\scriptsize S~\text{по Тёплицу}}{\dfrac{1}{b_n}\sum\limits_{j = 1}^n S_{j-1} a_j} \limn 0.$$
	\end{proof}
\end{lemma}

\begin{theorem}[УЗБЧ в форме Колмогорова-Хинчина]
	Пусть 
	\begin{enumerate}
	    \item случайные величины $\{\xi_n\}_{n \geqslant 1}$ независимы и $\forall n \hookrightarrow \D \xi_n < +\infty$;
	    \item числа $\{ b_n \}_{n \geqslant 1}$, $b_1 > 0$ и $b_n \uparrow +\infty$, такие что $\sum\limits_{n =1}^\infty \dfrac{\D \xi_n}{b_n^2} < + \infty$;
	    \item обозначим $S_n = \sum\limits_{i = 1}^{n} \xi_i$.
	\end{enumerate}  
	Тогда $\dfrac{S_n - \E S_n}{b_n} \xrightarrow[n \rightarrow + \infty]{\text{п.н.}} 0$.
	\begin{proof}
		Преобразуем:
		$$ \frac{S_n - \E S_n}{b_n} = \frac{1}{b_n} \sum\limits_{i = 1}^{n} b_i \cdot \frac{\xi_i - \E \xi_i}{b_i}.$$
		Обозначим $\eta_i = \dfrac{\xi_i - \E \xi_i}{b_i}$. Случайные величины $\eta_i$ независимы и  $\E \eta_i = 0$. Значит,  
		$$ \sum\limits_{i =1}^\infty \E \eta_i^2 = \sum\limits_{i = 1}^\infty \dfrac{\E( \xi_i - \E \xi_i)^2}{b_i^2} = \sum\limits_{i = 1}^\infty \dfrac{\D \xi_i}{b_i^2} < +\infty.$$
		Следовательно, по теореме Колмогорова-Хинчина о сходимости ряда $\sum\limits_{i=1}^n \eta_i$ сходится почти наверное. По лемме Кронекера последовательность 
		$$\dfrac{1}{b_n} \sum\limits_{i =1}^n b_i \cdot \dfrac{\xi_i - \E \xi_i}{b_i}$$ 
		сходится к нулю для всех $\omega$, для которых сходится ряд 
		$$\sum\limits_{i = 1}^\infty \dfrac{\xi_i - \E \xi_i}{b_i} = \sum\limits_{i = 1}^\infty \eta_i \text{ ---		сходится п.н.}$$
		Следовательно, 
		$$ \dfrac{1}{b_n} \sum\limits_{i =1}^n b_i \cdot \dfrac{\xi_i - \E \xi_i}{b_i} = \frac{S_n - \E S_n}{b_n} \xrightarrow[n \rightarrow \infty]{\text{п.н}} 0.$$
	\end{proof}
\end{theorem}

\begin{lemma}
	Пусть $\xi \geqslant 0$, $\E \xi < + \infty$, тогда 
	$$ \sum\limits_{n=1}^\infty \P (\xi \geqslant n) \leqslant \E \xi \leqslant 1 + \sum\limits_{n = 1}^\infty \P ( \xi \geqslant n).$$
	\begin{proof}
		\begin{multline*}
			\sum\limits_{n = 1}^\infty \P ( \xi \geqslant n) = 
			\sum\limits_{n = 1}^\infty \sum\limits_{k = n}^\infty \P (k \leqslant \xi \leqslant k + 1) = 
			\sum\limits_{k =1}^\infty \sum\limits_{n = 1}^{k} \P( k \leqslant \xi \leqslant k + 1) = \\ = 
			\sum\limits_{k = 0}^{\infty} k \cdot \P ( k \leqslant \xi \leqslant k + 1) = 
			\sum\limits_{k = 0}^\infty \E \big( k \cdot I (k \leqslant \xi  \leqslant k + 1) \big) = \\ = 
			\sum\limits_{k = 0}^\infty \E \big( \lfloor \xi \rfloor \big) \cdot I (k \leqslant \xi \leqslant k + 1 ) \leqslant
			\sum\limits_{k = 0}^\infty \E \big( \xi \cdot I ( k \leqslant \xi \leqslant k+ 1) \big) = \\ =
			\E \left( \xi \cdot \sum\limits_{k = 0}^\infty I (k \leqslant \xi \leqslant k + 1) \right) = \E \xi.
		\end{multline*}
		Верхнее неравенство доказывается аналогично.
	\end{proof}
\end{lemma}

\begin{definition}
	Случайные величины $\xi $ и $\eta$ одинаково распределены, если  $\forall x \hookrightarrow F_\xi (x) = F_\eta(x)$. Обозначают $\xi \overset{d}{=} \eta$.
\end{definition}

\begin{statement}
Если $\xi \overset{d}{=} \eta$, то $\forall g(x) \hookrightarrow \E g(\xi) = \E g(\eta)$.
\begin{proof}
	$\E g(\xi) = \int\limits_{\R} g(x) \, d F_\xi(x) = \int\limits_{\R} g(x) \, d F_\eta(x) = \E g(\eta)$.
\end{proof}	
\end{statement}

\begin{theorem}[УЗБЧ в форме Колмогорова]
	Пусть $\{\xi_n\}_{n \in \N}$~--- независимые одинаково распределенные случайные величины такие, что $\E | \xi_1| < + \infty$. 
	
	Тогда
	$$ \frac{\xi_1 + \ldots + \xi_n}{n} \xrightarrow[n \rightarrow  \infty]{\text{п.н.}} \E \xi_1.$$
	\begin{proof}
		Поскольку $\E | \xi_1| < +\infty$, то по предыдущей лемме $\sum\limits_{n = 1}^\infty \P \big( |\xi_1| \geqslant n \big) < +\infty$. Так как $\xi_1 \overset{d}{=} \xi_n$, то $\sum\limits_{n = 1}^\infty \P \big( |\xi_n| \geqslant n \big) < + \infty$, следовательно, по лемме Бореля-Кантелли $\P \Big( \big\{ |\xi_n| \geqslant n \big\}~\text{б.ч.} \Big) = 0$. То есть c вероятностью 1 случается конечное число $\big\{ |\xi_n| \geqslant n \big\}$. Обозначим $\tilde \xi_n \equiv \xi_n \cdot I \big\{ | \xi_n | \leqslant n \big\}$. Тогда с вероятность 1 $\xi_n = \tilde \xi_n$ кроме конечного числа $\xi_n$. Пусть $\E \xi_i = 0$, если это не так, то $\eta_i = \xi_i - \E \xi_i$. Получаем, что 
		$$ \P \left( \frac{\xi_n + \ldots + \xi_n}{n} \rightarrow 0 \right) = \P \left( \frac{\tilde \xi_1 + \ldots + \tilde \xi_n}{n} \rightarrow 0 \right).$$
		Рассмотрим
		$$ \E \tilde\xi_n = \E \Big( \xi_n \cdot I \big( | \xi_n | \leqslant n \big) \Big) = \E \Big( \xi_1 \cdot I \big( | \xi_1| \leqslant n \big) \Big) \rightarrow \E \xi_1 = 0$$
		по теореме Лебега о мажорируемой сходимости, поскольку
		$$\Big| \xi_1 \cdot I \big( |\xi_1| \leqslant n \big) \Big| \leqslant \xi_1~~\text{и}~~\xi_1 \cdot I \big( |\xi_1| \leqslant n \big) \xrightarrow[n \rightarrow \infty]{\text{п.н.}} \xi_1.$$
		По лемме Тёплица
		$$ \frac{1}{n} \sum\limits_{i = 1}^{n} \E \tilde\xi_i \rightarrow \E \xi_1 = 0 \quad \Rightarrow \quad 
		\dfrac{1}{n}\sum\limits_{i = 1}^n \tilde\xi_i \xrightarrow[n \rightarrow \infty]{\text{п.н.}} 0 \quad \Leftrightarrow \quad
		 \dfrac{1}{n}\sum\limits_{i = 1}^n \left(\tilde\xi_i - \E \tilde\xi_i \right) \xrightarrow[n \rightarrow \infty]{\text{п.н.}} 0.$$ 
		Обозначим $\overline \xi_n = \tilde \xi_n - \E \tilde \xi_n$. По лемме Кронекера, если сходится $\sum\limits_{k = 1}^\infty \dfrac{\overline \xi_k}{k}$ на каком-то $\omega$, то $\dfrac{1}{n} \sum\limits_{k = 1}^n k \cdot \dfrac{\overline \xi_k}{k} \limn 0$ на том же $\omega$.  Проверим, что $\sum\limits_{k = 1}^\infty \dfrac{\overline \xi_k}{k}$ сходится почти наверное. По теореме Колмогорова-Хинчина достаточно показать, что $\sum\limits_{k = 1}^\infty \dfrac{\E \left(\overline\xi_k\right)^2}{k^2} < + \infty$.
		\begin{multline*}%плохая верстка, но мне уже лень
			\sum\limits_{k = 1}^\infty \dfrac{\E \left( \overline\xi_k \right)^2}{k^2} = 
			\sum\limits_{k = 1}^\infty \dfrac{\E \left( \tilde\xi_k - \E \tilde\xi_k \right)^2}{k^2} \leqslant
			\sum\limits_{k = 1}^{\infty} \dfrac{\E \left( \tilde\xi_k\right)^2}{k^2} = 
			\sum\limits_{k= 1}^\infty \dfrac{1}{k^2} \cdot \E\Big( \xi_k^2 \cdot I \big( | \xi_k| \leqslant k \big) \Big) = \\ = 
			\sum\limits_{k= 1}^\infty \dfrac{1}{k^2} \cdot \E\Big( \xi_1^2 \cdot I \big( | \xi_1| \leqslant k \big) \Big) = 
			\sum\limits_{k= 1}^\infty \dfrac{1}{k^2} \cdot \E\left( \xi_1^2 \cdot \sum\limits_{n = 1}^k I \big( n -1 < | \xi_1| \leqslant n \big) \right) = \\ = 
			\sum\limits_{n = 1}^\infty \E \Big( \xi^2 \cdot I \big( n - 1 < |\xi_1| \leqslant n \big) \Big) \cdot \underbracket[0.5pt]{\sum\limits_{k = n}^\infty \dfrac{1}{k^2}}_{\leqslant 2/n} \leqslant
			\sum\limits_{n = 1}^\infty \dfrac{2}{n} \cdot \E \Big( \xi_1^2 \cdot I \big( n - 1 < | \xi_1| \leqslant n \big) \Big) \leqslant \\
			\text{Так как $|\xi_1| \leqslant n$, то заменим одну $\xi$ на n:} \\
			\leqslant 
			2 \sum\limits_{n =1}^\infty \E | \xi_1| \cdot I \big( n - 1 < |\xi_1| \leqslant n \big) \overset{\text{по т. Беппо-Леви}}{=} 
			2 \E | \xi_1| \sum\limits_{n = 1}^\infty I \big( n - 1 < |\xi_1|  \leqslant n \big) = \\ =
			 2 \E|\xi_1| < + \infty.
		    \end{multline*}
	\end{proof}
\end{theorem}

\begin{theorem}[Беппо-Леви]
	Пусть $\{ \xi_n \}_{n \geqslant 1}$~--- случайные величины, $\forall n \hookrightarrow \xi_n \geqslant 0$.
	
	Тогда $\E \sum\limits_{n = 1}^\infty \xi_n = \sum\limits_{n = 1}^\infty \E \xi_n$.
	\begin{proof}
		Пусть $S_n = \sum\limits_{k = 1}^n \xi_k$, тогда $S_n \uparrow S = \sum\limits_{k = 1}^\infty \xi_k$. По теореме о монотонной сходимости $\E \sum\limits_{k = 1}^n \xi_k \limn \E \sum\limits_{k = 1}^\infty \xi_k$, следовательно,
		$$ \E \sum\limits_{k = 1}^n \xi_k = \sum\limits_{k = 1}^n \E \xi_k \uparrow \E \sum\limits_{k = 1}^\infty \E \xi_k.$$
	\end{proof}	
\end{theorem}

\begin{theorem}[о монотонной сходимости][б/д]
	Пусть $\{\xi_n\}_{n \geqslant 1}, \xi, \eta$~--- случайные величины, тогда
	\begin{enumerate}
		\item {Если $\xi_n \uparrow \xi$ почти наверное и $ \forall n \in \N \hookrightarrow \xi_n \geqslant \eta, \E \eta > - \infty$, то $\E \xi = \lim\limits_{n \rightarrow \infty} \E \xi_n$.}
		\item {Eсли $\xi_n \downarrow \xi$ почти наверное и $ \forall n \in \N \hookrightarrow \xi_n \leqslant \eta, \E \eta < + \infty$, то $\E \xi = \lim\limits_{n \rightarrow \infty} \E \xi_n$.}
	\end{enumerate}
\end{theorem}

\begin{lemma}[Фату]
	Пусть $\{ \xi_n \}_{n \geqslant 1}$ и $\eta$~--- случайные величины, $\E | \eta| < + \infty$, тогда 
	\begin{enumerate}
		\item { Если $\forall n \hookrightarrow \xi_n \geqslant \eta$, то $ \varliminf\limits_{n \rightarrow \infty} \E \xi_n \geqslant \E \varliminf\limits_{n \rightarrow \infty} \xi_n$.}
		\item { Если $\forall n \hookrightarrow \xi_n \leqslant \eta$, то $ \varlimsup\limits_{n \rightarrow \infty} \E \xi_n \leqslant \E \varlimsup\limits_{n \rightarrow \infty} \xi_n$.}
		\item { Если $\forall n \hookrightarrow |\xi_n| < \eta$, то $ \E \varliminf\limits_{n \rightarrow \infty} \xi_n \leqslant \varliminf\limits_{n \rightarrow \infty} \E \xi_n \leqslant \varlimsup\limits_{n \rightarrow \infty} \E \xi_n \leqslant \E \varlimsup\limits_{n \rightarrow \infty} \xi_n$.}
	\end{enumerate}
	\begin{proof}
		(1) Обозначим $\psi_n = \inf\limits_{k \geqslant n} \xi_k$. Очевидно, $\psi_n \uparrow \varliminf\limits_{n \rightarrow \infty} \xi_n$. кроме того $\psi_n \geqslant \eta$, следовательно, по теореме о монотонной сходимости $ \lim\limits_{n \rightarrow \infty} \E \psi_n = \E \varliminf\limits_{n \rightarrow \infty} \xi_n$. Рассмотрим 
		$$\E \varliminf\limits_{n \rightarrow \infty} \xi_n = \lim\limits_{n \rightarrow \infty} \E \psi_n = \varliminf\limits_{n \rightarrow \infty} \E \psi_n \overset{\text{т.к.}~\psi_n \leqslant \xi_n}{\leqslant} \varliminf\limits_{n \rightarrow \infty} \E \xi_n.$$
		
		Второе равенство -- в силу существования предела существует и нижний предел.
		
		(2) Следует из пункта (1) заменой $\xi'_n = - \xi_n$.
		
		(3) Следует из (1) и (2).
	\end{proof}
\end{lemma}

\begin{theorem}[Лебега о мажорируемой сходимости]
	Пусть $\xi_n \xrightarrow{\text{п.н.}} \xi, |\xi| \leqslant \eta, \E \eta < + \infty$.
	
	Тогда $\E \xi_n \limn \E\xi$ и $\E | \xi_n - \xi| \limn 0$ (не требуем независимости!).
	\begin{proof}
		Заметим, что $\xi \overset{\text{п.н.}}{=} \lim\limits_{n \rightarrow \infty} \xi_n = \varliminf\limits_{n \rightarrow \infty} \xi_n = \varlimsup\limits_{n \rightarrow \infty} \xi_n$. По пункту (3) леммы Фату
		\begin{multline*}
			\E \xi = \E \varliminf\limits_{n \rightarrow \infty} \xi_n \leqslant \varliminf\limits_{n \rightarrow \infty} \E \xi_n \leqslant \varlimsup\limits_{n \rightarrow \infty} \E \xi_n \leqslant  \E \varlimsup\limits_{n \rightarrow \infty} \xi_n  = \E \xi \quad \Rightarrow \quad \E \xi = \lim\limits_{n \rightarrow \infty} \E \xi_n.
		\end{multline*}
		Конечность $\E \xi$ следует из того, что $|\xi| < \eta$ почти наверное, следовательно, так как $\E \eta < + \infty$, то $\E |\xi| \leqslant \E | \eta | < + \infty$.
		
		Докажем $L_1$-сходимость. Возьмем $\psi_n = |\xi_n - \xi|$. Тогда $|\psi_n| \leqslant 2 \eta$ почти наверное и $\psi_n \xrightarrow{\text{п.н.}} 0$, следовательно, $\E \psi_n \limn 0$ по теореме Лебега.
	\end{proof}
\end{theorem}

\subsection{Сходимость в $L_2$}
Введем пространство $L_2 = L_2(\Omega, \F, \P) = \{ \xi: \E \xi^2 < + \infty \}$. Это минимальное пространство, так как $\E ( a \xi + b \eta)^2 \leqslant 2 a^2 \E\xi^2 + 2b^2\E\eta^2$. \\

Основное неравенство: $(x+y)^2 \leqslant 2x^2 + 2y^2$.\\

Норма $ \| \xi \| = \sqrt{\E \xi^2}$;  скалярное произведение $(\xi, \eta) = \E \xi\eta$.
\begin{lemma}
	Пусть $\xi_n \xrightarrow{L_2} \xi$, $\forall n \hookrightarrow \xi_n \in L_2$. Тогда
	\begin{enumerate}
		\item $\xi \in L_2$,
		\item $\E\xi_n \limn \E \xi$,
		\item $\E \xi_n^2 \limn \E\xi^2$,
		\item {если $\eta_n \xrightarrow{L_2} \eta$, $\forall n \hookrightarrow \eta_n \in \L_2$, то $(\xi_n, \eta_n) \limn (\xi, \eta)$.}
	\end{enumerate}
	\begin{proof}
		Докажем первый пункт леммы:
		$$\E \xi^2 = \E ( \xi - \xi_n + \xi_n)^2 \leqslant \underbracket[0.5pt]{2\E( \xi - \xi_n)^2}_{\rightarrow 0} + \underbracket[0.5pt]{2\E \xi_n^2}_{< + \infty} < +\infty.$$
		Перейдем ко второму пункту. Если $\E \xi^2 < + \infty$, то $\E | \xi | = \E | \xi| \cdot 1$, а по неравенству Коши-Буняковского это меньше или равно, чем $\sqrt{\E \xi^2 \cdot \cancelto{1}{\E 1^2}} < + \infty$. Осталось заметить, что $\big| \E ( \xi_n - \xi ) \big| \leqslant \E | \xi_n - \xi | \leqslant \sqrt{\E (\xi_n - \xi)^2 \cdot \E 1^2} \limn 0$.
		Пункт 3. 
		\begin{multline*}
			\E ( \xi_n^2 - \xi^2) = 
			\E ( \xi_n + \xi)(\xi_n - \xi) \leqslant \sqrt{ \E (\xi_n + \xi)^2 \cdot \E (\xi_n - \xi)^2} \leqslant \\ \leqslant
			 \sqrt{\big( \underbracket[0.5pt]{2 \E( \xi_n - \xi)^2}_{\rightarrow 0} + \underbracket[0.5pt]{8\E \xi^2}_{=\const} \big) \cdot \underbracket[0.5pt]{\E (\xi_n^2 - \xi^2)}_{\rightarrow 0}} \limn 0.
		\end{multline*}
		Остается доказать четвертый пункт леммы:
		\begin{multline*}
			\E ( \xi_n \eta_n - \xi \eta) = 
			\E( \xi_n \eta_n - \xi_n \eta) + \E( \xi_n \eta - \xi \eta) \leqslant  \\ \leqslant
			\sqrt{\E\xi^2 \cdot \underbracket[0.5pt]{\E (\eta_n - \eta)^2}_{\rightarrow 0}} + \sqrt{\E \eta^2 \cdot \underbracket[0.5pt]{\E(\xi_n - \xi)^2}_{\rightarrow 0}} \limn 0.
		\end{multline*}
	\end{proof}
\end{lemma}